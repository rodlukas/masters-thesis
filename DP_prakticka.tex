\chapter{Sběr požadavků a analýza}\label{chap:sberpozadavkuaanalyza}

\section{Požadavky}

Následuje seznam funkčních, nefunkčních a vyřazených požadavků na rozšíření aplikace. Funkční a nefunkční požadavky mají pro následnou práci přiřazeny unikátní identifikátory ve tvaru \enquote{F<číslo>}/\enquote{N<číslo>} (kde písmeno značí \textbf{f}unkční/\textbf{n}efunkční požadavek). Hvězdičkou jsou dále označeny požadavky, které vychází z plánovaných rozšíření v rámci bakalářské práce uvedených v sekci~\ref{sec:planrozsirenibp}. Některé plánované změny z této kapitoly byly vyřazeny a je jim věnována poslední podsekce~\ref{subsec:vyrazenepozadavky}.

\subsection{Funkční požadavky}

\begin{itemize}
    \item \textbf{F1 -- evidování zájemců o kurz *:} vzhledem k plnému obsazení všech termínů během týdne je třeba evidovat zájemce o kurzy -- ať už jednotlivce, či zájemce o skupiny -- každý klient může mít zájem o daný kurz nejvýše jednou, je třeba také evidovat k tomuto zájmu text formou poznámky, datum přidání zájemce do evidence a kurz, o který má zájem,
    \item \textbf{F2 -- kontrola časového konfliktu lekcí *:} současná verze aplikace nijak neřeší časové konflikty lekcí, je třeba zakázat možnost jakéhokoliv překryvu lekcí vzhledem k datu, času a jejich délce, toto se netýká zrušených lekcí, které budou pro řešení konfliktů ignorovány,
    \item \textbf{F3 -- vylepšení předplacených lekcí *:} stávající způsob evidence předplacených lekcí není dostačující -- pro jednotlivce je třeba předplacené lekce přidávat po jednom (klienti si často předplácí více lekcí dopředu), pro skupiny je evidování ještě horší, protože každý člen obvykle platí jinak a na odlišné časové období (případně vždy jen jednu lekci) a prakticky se nedá tato evidence předplacených lekcí ručně udržovat, je to příliš složité,
    \item \textbf{F4 -- vyhledávání klientů *:} v aplikaci je mnoho klientů a je časově náročně vždy v seznamu vyhledávat příslušného klienta, je třeba zavést možnost vyhledávání v klientech, a to z jakéhokoliv místa aplikace, také je třeba, aby vyhledávání bralo v potaz možné překlepy lektorky v zadávaném výrazu k vyhledávání a také možný překlep ve jménu uloženého klienta, vyhledávalo by se jen mezi aktivními klienty (viz požadavek F6),
    \item \textbf{F5 -- přepracování formuláře pro lekce:} současný formulář není úplně přehledný, pokud má skupina více než 2 členy, je potřeba neustále posouvat obsahem, protože se nevejde na monitor, je třeba jej kompletně přepracovat dle konzultace s lektorkou,
    \item \textbf{F6 -- zavedení aktivních a neaktivních klientů a skupin:} v evidenci je mnoho klientů a skupin, kteří aktuálně nechodí, ale např. za rok budou opět chodit, je tedy třeba umožnit skrytí všech klientů/skupin, kteří aktuálně na lekce nedochází a k těmto skrytým (neaktivním) klientům/skupinám umožnit přístup, ve výchozím zobrazení ale ukazovat jen aktivní,
    \item \textbf{F7 -- nastavitelná délka kurzů:} současná verze aplikace automaticky předvyplní dobu trvání lekce v závislosti na tom, zda je skupinová (45~min.) či pro jednotlivce (30~min.) -- to není dostačující, protože např. lekce některých kurzů trvají vždy 45~min. nehledě na počet členů -- je tedy třeba umožnit u kurzu evidovat dobu trvání pro jednotlivce, a tu pak ve výchozím stavu jednotlivcům dávat, pro skupiny stále stačí jedna výchozí hodnota (45~min.),
    \item \textbf{F8 -- zobrazit poslední transakce na bankovním účtu:} na hlavní stránce je třeba mimo dnešních lekcí třeba zobrazit aktuální zůstatek na bankovním účtu projektu (Fio banka) a transakce za poslední 3~týdny,
    \item \textbf{F9 -- změny účastníků skupinových lekcí:} někdy se stává, že klient v průběhu kurzu opustí skupinu, v evidenci je již naplánováno několik lekcí dopředu a všech se účastní -- je třeba umožnit při úpravě lekce projevit tyto změny členů skupiny do účastníků dané lekce (v současné době nečlen skupiny zůstává účastníkem lekce a lektorka musí každou lekci ručně smazat a vytvořit znovu bez nečlenů, což je velmi nepohodlné),
    \item \textbf{F10 -- automatické přidání předplacené lekce:} při omluvě klienta/zrušení ze strany lektorky je třeba automaticky u klienta/skupiny zaznamenat, že mají jeden předplacený termín navíc (v případě, že daný klient měl zaplaceno), v případě jednotlivců je třeba také automaticky zaevidovat, za který den je tato náhradní lekce vytvořena,
    \item \textbf{F11 -- efektivnější práce v rámci aplikace:} v současné verzi aplikace musela lektorka pro často prováděné činnosti provádět zbytečně mnoho kroků navíc, což činilo příslušné činnosti náročnějšími a snadno se udělala chyba a ztrácel čas -- na základě dalších analýz byly zjištěny problémové oblasti, které vyžadují přepracování: umožnit úpravu lekce (pro jednotlivce i skupiny) z diáře a přehledu, umožnit úpravu klienta/skupiny přímo v kartě klienta/skupiny, umožnit přidání lekce (pro jednotlivce i skupiny) z diáře a přehledu (a to jak s příslušným datem, u kterého bude toto tlačítko, tak i obecně s jakýmkoliv datem), umožnit přidat klienta přímo při přidávání skupiny/zájemce (funkcionalita zájemce implementována v rámci F1),
    \item \textbf{F12 -- evidování barev kurzů:} je třeba umožnit přiřazení barvy každému kurzu a tuto barvu použít pro rozlišení kurzů napříč celou aplikací, kdekoliv se vyskytuje název kurzu -- lektorka potřebuje okamžitě rozlišit kurzy a mít možnost na první pohled např. v diáři vidět díky barvě, kterého kurzu se barva týká (kurzy mají v rámci propagačních materiálů své ustálené barvy),
    \item \textbf{F13 -- automatické předvyplnění údajů lekce:} pokud má klient/skupina nějakou historii v evidenci, při přidávání lekce je třeba na základě této historie předvyplnit vhodnými hodnotami datum, čas a kurz lekce -- je tedy třeba vhodně zvolit lekci klienta z jeho historie, podle které budou tyto údaje vypočteny (tedy hodnota data a času bude o týden později oproti zvolené lekci, kurz bude tentýž) -- smyslem je těmito výchozími hodnotami vystihnout co nejvíce případu tak, aby lektorka musela tyto hodnoty co nejméně často upravovat,
    \item \textbf{F14 -- upozornění na ztrátu dat formuláře:} při vyplňování formuláře lektorka občas omylem formulář zavře a přijde o úpravy, je třeba zavést ochranu, která tomuto zabrání (a zároveň ale nebude zbytečně upozorňovat na ztrátu dat, když k žádné změně nedošlo),
    \item \textbf{F15 -- vylepšení chybových hlášení:} chybová hlášení jsou někdy málo podrobná, zobrazují se krátce, případně dojde k neošetřené chybě na serveru a klientská část pak nedokáže korektně uživateli popsat, kde je problém,
    \item \textbf{F16 -- titulky stránek:} každá stránka by měla mít v prohlížeči svůj titulek, v současné době má každá stránka stejný titulek a lektorka tak nemá možnost jednoduše rozlišit, který panel má otevřenou kterou stránku aplikace,
    \item \textbf{F17 -- nastavitelné vlastnosti stavů účasti:} některé stavy účasti mají pro některé výpočty v rámci aplikace speciální význam (např. omluvené lekce se nezapočítávají do počtu absolvovaných lekcí klienta, další stav znamená, že klient dorazí a tento stav je zároveň výchozí), toto svázání stavu účasti s významem je ale založeno na názvu stavu účasti (a pevně nadefinováno v kódu), lektorka si chce ale název příslušných stavů účasti sama měnit a je třeba pro to zavést příslušné možnosti,
    \item \textbf{F18 -- omezení a validace hodnot:} je třeba provést revizi stávajících omezení na jednotlivé hodnoty v rámci aplikace (API a databáze) i klientské části, zavést nová omezení pro nové funkční požadavky a zdokumentovat všechna aktuální omezení a validace prováděná nad celou doménou,
    \item \textbf{F20 -- automatické zrušení lekce:} pokud nikdo z účastníků nemá dorazit (všichni jsou omluveni), lekce má být automaticky zrušena,
    \item \textbf{F21 -- zobrazení zrušených lekcí:} v přehledu na hlavní stránce a v diáři je třeba zobrazit i zrušené lekce, v současné verzi se nezobrazují (to byl původní požadavek, ale nakonec se ukázal jako nesprávný a lektorka lekce potřebuje vidět nejen v kartě klienta).
\end{itemize}

\subsection{Nefunkční požadavky}

\begin{itemize}
    \item \textbf{N1 -- dokumentace:} dokumentace v kódu pro serverovou i klientskou část, dokumentace API,
    \item \textbf{N2 -- testování *:} aplikace prakticky neobsahuje žádné testy (jen několik základních \enquote{smoke} testů), je třeba zavést API a UI (e2e) testy pro klíčové části aplikace a průchody v aplikaci,
    \item \textbf{N3 -- zavedení nástrojů pro usnadnění vývoje a údržby:} je třeba zavést nástroje pro monitorování chyb v aplikaci, správu logů (vyhledávání, ukládání), statické typování a analýzu kódu -- k tomu byla provedena v teoretické části rešerše v kapitole~\ref{chap:nastrojeprousnadnenivyvojeaudrzby},
    \item \textbf{N4 -- revize bezpečnosti:} je třeba provést kompletní revizi aplikace z hlediska bezpečnosti a opravit případné slabiny, problémy či zranitelnosti,
    \item \textbf{N5 -- zvýšení použitelnosti a přístupnosti:} z hlášení lektorky a také z vlastní analýzy vyplývá, že je třeba se zaměřit na opravení problémů s použitelností a přístupností,
    % todo vysvetlit pristupnost
    \item \textbf{N6 -- rychlejší získání dat z API:} pokud klientská část má někde zobrazit více dat, požadavek někdy probíhá enormně dlouho a dokonce může být ze strany Heroku pro dlouhou prodlevu zastaven a lektorka si data nezobrazí -- je třeba nalézt úzké hrdlo aplikace, které toto zpomalení v jednotkách případů (když je hodně dat) zpomaluje a napravit.
\end{itemize}

% todo BP: migrace na novy react, react context

\subsection{Vyřazené požadavky}\label{subsec:vyrazenepozadavky}
\begin{itemize}
    \item \textbf{evidence pomůcek a učebnic *:} v rámci projektu ÚP již existuje starší aplikace na míru, která tuto funkcionalitu dostatečně řeší a zatím není potřeba toto řešení integrovat do jednotného řešení,
    \item \textbf{offline přístup a SSR *:} co se týče SSR, načítání aplikace je dostatečně rychlé a není zde tedy potřeba prozatím SSR řešit, řešení offline režimu zatím není ze strany lektorky požadováno (stačí jí stávající řešení).
\end{itemize}

\section{Procesy a entity}
je treba zjistit a uvest informace k novym entitam a procesum, napr. jak funguji zajemci o kurz, jak evidovani predplacenych lekci.. aby se na zaklade toho mohl udelat navrh


\chapter{Návrh}
navrh upravene DB a API a pripadne dalsich veci

* pokročilejší evidence předplacených kurzů (skupiny - citac, indiv - vyber poctu), 
* api krestni jmeno klienta reprezentovane "firstname"
* vice default hodnot, upravy diagramu
* [api] poradove cislo lekce ma rozumnejsi nazev number (byl count)

N2: 
* vylepseni vsech testu - neprobihala kontrola zobrazenych dat ve formulari pred upravou
* zrychleni testu upravy klientu
* vylepseni potvrzovani formulare pro lepsi zachyceni chyb
* oprava náhodně občas nefungujících testů na CI (po každém scénáři je smazána localstorage) (\#64)

F6: 
* pri uprave/pridani skupiny/klienta dojde automaticky k prepnuti na zalozku aktivni/neaktivni podle toho, do ceho jsme pridavali

F8:
* zobrazují se transakce za posledních 14 dnů, FIO omezuje dotazy na API na 30 s, takže server cachuje odpověď banky rovnou 60 s, zobrazuje se i aktuální zůstatek + barva je podle toho, zda je na účtu dostatek peněz na zaplacení nájmu (zelená, resp. červená s upozorňujícím vykřičníkem s dalšími informacemi), po 60 s možnost výpis obnovit, případně lze jít do bankovnictví, opravuje \#50
* cache, upozorneni na najem

F18:
* pokročilá validace duplicit apod.
* zruseni noveho omezeni "neaktivní klienty nelze přiřadit do skupin (UI je ani nezobrazí ve výběru)" - neaktivní klienti již mohou být členové skupiny
* velka pocatecni pismena jmena klienta i pro api, presun transformaci field z modelu do serializeru
* viz EA docs
* js zvyrazneni neaktivity skupin a klientu napric aplikaci https://github.com/rodlukas/UP-admin/issues/85

F15:
* frontend odolny proti padu diky ErrorBoundary
* drf zpracovavat ProtectedError https://github.com/rodlukas/UP-admin/issues/70
* podrobnejsi chyby z api
* osetreni v api (IO...)
* notifikace neprekryvaji menu
* osetreni vstupu (vcetne konecne plne funkcniho API)
* js vylepseni notifikaci - notifikace doplnene o nadpis a ikonu, sjednoceny font, komponenta pro obsah notifikace, delsi doba zobrazeni error notifikace
* podrobnější chybové hlášky (např. při ProtectedError, když se uživatel pokouší mazat instanci, na které závisí jiné instance s chráněným příslušným FK; podrobnější info o časovém konfliktu...)

N4:
* zavedeni referrer-policy (same-origin), bezpecnejsi konfigurace lokalniho prostredi
* oprava zabezpeceni aplikace + automaticke presmerovani na https + HSTS
* mnoho promennych prostredi → ve všech prostředích se pracuje stejným způsobem s příslušnými proměnnými prostředí, je čistější nastavení Djanga a umožňuje díky .env souboru (je v .gitignore) použít i lokální proměnné prostředí (jinak těžko univerzálně nastavitelné i v rámci IDE) + odstranění zašifrovaných proměnných v konfiguráku travisu a přesun přímo do repo
* robots.txt, pak nahrazeni robots.txt meta tagem pro uplny zakaz cehokoliv robotum
* oprava zacykleni aplikace pri expirovanem tokenu (pri otevreni aplikace se zacyklila a zpusobila velke vytizeni CPU)
* "noopener noreferrer" z důvodu bezpečnosti
* nastavení DEBUG = False pro Django produkci, opravuje \#23 
* úprava práce s tokeny, zašifrování, refresh FontAwesome tokenu, opravuje mj. \#28
* contextapi pro prihlasovani - zmeny viz https://github.com/rodlukas/UP-admin/releases/tag/0.8

F17:
* stav omluven a zruseno uz neni napevno

N3:
* refaktoring
* js komponenta pro zobrazeni lekci https://github.com/rodlukas/UP-admin/issues/71
* template strings js
* python refaktoring https://github.com/rodlukas/UP-admin/issues/93
* zavedeni prvnich hooku, drobna vylepseni komponent - prechod na funkcionalni komponenty, upravy komponent (napr. zbytecne pouzivaly stav nebo se zbytecne casto updatovaly, ikdyz to nebylo potreba) + **odstranění zbytečného stavu odstranilo i chybu, která byla už od počátku v aplikaci a způsobovala občasné nefungování komponenty s aktuálním stavem účasti - zobrazil se jiný stav, než byl skutečný, v závislosti na pořadí obdržení odpovědí od serveru, to se začalo víc projevovat při nedávných změnách architektury reactu**
* opravy nekonzistentnich stavu
* analyza pruchodu aplikaci (google analytics)
* opravy nevalidniho html a css (diky tomu lepsi zarovnani tlacitek)
*  js useEffect - exhaustive deps https://github.com/rodlukas/UP-admin/issues/96
* prechod na TS: dalsi zmeny viz https://github.com/rodlukas/UP-admin/releases/tag/1.0.0
* Minifikace HTML
* oprava práce se setState - zbytečně se přes const state = this.state nastavoval pak ve setState celý nový stav, což je zbytečné a navíc to může způsobovat problémy při dalším volání setState (které když proběhne před tímto, může přijít v niveč, protože jej nahradí tato kompletní změna stavu, ačkoliv se má měnit jen malá část stavu komponenty) 

N5:
* mene pozadavku - react context api (attendancestates + nově také viditelné kurzy, aktivní klienti/skupiny - vše napříč aplikací) -> TÍM PÁDEM POTŘEBA PŘEJÍT NA NOVÝ REACT -> je tedy potřeba provést i migraci na nový lifecycle reactu, vice viz https://github.com/rodlukas/UP-admin/releases/tag/0.8
* zvýšení usability/acessibility https://github.com/rodlukas/UP-admin/issues/27
* js react-select escape zavře i modal okno https://github.com/rodlukas/UP-admin/issues/84, oprava chovani ESC v CustomInput a v react-select 
* opravy zalamovani textu
* procházení diářem šipkami na klávesnici + request až po sekundě - js pozdrzeni requestu DashboardDay v diari pri preklikavani tydnu 
* všude načítání, když jsou nějaké požadavky
* vylepseni responzivity
* js plynule posouvani v modalnim okne v Safari (iOS)
* využívání react-selectu napříč aplikací (výběr kurzu)
* klikani na labely vsude funguje
* redesign vsech stranek pro lepsi konzistenci, srozumitelnost a pochopeni (predelany diar, zajemci, lekce v karte...)
* procházení diářem šipkami na klávesnici 
* Pro zrychlení načítání celé aplikace se používá lazy loading React.lazy + React Suspense - umozni kratsi prvni nacteni appky, zalozene na react-routeru + reseni nastaleho problemu -- reseni Error: Loading chunk 11 failed. https://github.com/rodlukas/UP-admin/issues/92
* optimalizace frontendu - odstraneni komponent definovanych primo v render - mohlo zpusobit potize s vykonem i nechtenym prekreslovanim vnorenych komponent
* favicony 
* Loading: upozorneni na presprilis dlouhe nacitani, indikace práce při delším načítání než 5 sekund
* [js] migrace title na tooltips
* zobrazeni nacitani po submitu formulare primo v tlacitku + pred nactenim formulare disabled submit tlacitka
* zvyrazneni povinnych poli ve formularich
* pochopitelnejsi nadpis pri vyberu klienta pro pridani lekce 
* zobrazeni jednotky (minut) ve formulari u trvani
* srozumitelnejsi chybova hlaseni s popisem mozneho reseni 
* doplneni vice title a pak prechod na tooltips kvuli telefonum
* pridani kontroly pravopisu do inputu a textarea
* razeni zajemcu podle prijmeni a jmena, viz https://github.com/rodlukas/UP-admin/issues/54
* tlačítka pro přepínání týdnu v diáři už neodskakují podle délky data v titulku
* placeholder date a time input polí (pro IE/macOS, kteří input date/time nepodporují)
* utils funkce pro práci s částkami v CZ formátu (nehledě na jazyk, při EN jazyku v OS byly formáty jiné, což je nežádoucí)
* na iOSu už uživatelské jméno nebude při psaní začínat velkým písmenem (autoCapitalize)
* úpravy autofocusu formulářů - autofocus na přihlašovací formulář (stačí načíst stránku a kliknout Enter a dojde k přihlášení, není potřeba klikat do formuláře/na tlačítko), autofocus na klienta u zájemců, povolení autofocusu na celé Modal okno s formulářem pro lekce (klávesou Tab se dá rovnou pohybovat ve formuláři)

LGTM: 

[js] odstraneni unused promenne v Applications
[js] oprava spatneho nazvu promenne v inicializaci ErrorBoundary
[js] oprava potencialnich chyb zpusobenych nespravnou aktualizaci stavu
[python] odstraneni vsech import *
[python] optimalizace importu
[js] optimalizace importu
[js] oprava primeho zapisovani do stavu v Card
[js] konzistentni stav PrepaidCounters
[js] konzistentni stav at\_state ve FormLecture



N6:
* DRF-JWT problémy, dotazy na SQL DB ikdyž nemají být - https://github.com/rodlukas/UP-admin/issues/51 - přechod na jinou knihovnu a zde PR na překlad do CZ
* pomalé SQL dotazy- obri optimalizace dotazu na DB (>4x zrychleni) diky DJDT, pokrocile optimalizace
* odstraneni zbytecne prace s DB napric api - zrychleni - commit https://github.com/rodlukas/UP-admin/commit/ba12eea6be642d0c56629ac607fb1f8ffab267f7
* vice info viz https://github.com/rodlukas/UP-admin/releases/tag/0.9.0

\chapter{Implementace}
součástí sekce bude i sypani popela na hlavu za chyby v bakalarce - popisu co vsechno se delo, co bylo za problemy (napr bad patterny v reactu, coz zpusobilo ruzne problemy)

PR:
* chyby při práci s modálními okny https://github.com/rodlukas/UP-admin/issues/95 - issue reactstrapu, vyjde nova verze
* react fontawesome PR pro vylepseni typescript podpory (ID atribut..)
* reactstrap tooltip ios fix

\section{Implementace požadavků}
IMPLEMENTACE POZADAVKU z analyzy vyse

* zavedeni yarn kvuli rychlosti
* pipfile
* zobrazení tel. čísel klientů v zájemcích o kurz (\#61)
* oprava logiky výpočtu "příště platit" - nebral v úvahu předplacené lekce (https://github.com/rodlukas/UP-admin/commit/e84aa4eec9bb93323a097830dd1416204658c87a)
* lekce je automaticky zrušená na frontendu i backendu když jsou všichni účastníci omluveni, na to je příslušně uživatel upozorněn ve formuláři (a zároveň zrušení nejde měnit, protože by backend lekci stejně tak či tak zrušil)
* rozdělení CSS stylů ke komponentám
* asynchronní update všech dní v týdnu v diáři při nějaké změně (ca53c16), opravuje \#19
* zjednodušení kódů a struktury souborů, refactoring, odstranění všech zbytečných getDerivedStateFromProps (nahrazeno např. componentDidUpdate)
* migrace na nový životní cyklus komponent Reactu (503b9ac)
* funkční zobrazení vývojové verze na jiném zařízení v síti
* oprava chyby způsobující nekorektní zobrazení lekcí v předchozím dnu - když např. byla lekce v 1 h ráno, v diáři se ukázala v předchozím dni jako poslední → porovnávávání datumu s TZ s datumem bez TZ → vyřešeno použitím "\_\_date" v querysetu (3da9db1)
* oprava chybné práce s datumem v diáři (JS) - pokud se zadala URL s datumem, kde den měl číslo "31", došlo k "přesunu do minulosti" (číslo dne se změnilo na "1") - v důsledku toho se v aplikaci nedalo dostat do roku 2019, protože 31.12.2018 bylo pondělí a další týden se přepnul na listopad
* 05-2019 prechod na nwb
* úprava stavových kódu API pro bankovnictví (\#55) - už vrací jen 200/500, ostatní chyby jsou zahrnuty v 500 a podrobnější informace jsou přiloženy do JSONu rovnou na serveru (tedy na frontendu není žádná logika navíc)
* úpravy a opravy API - na backendu nechybí žádná logika, která doteď mohla být třeba i jen na frontendu - už fungují všude PATCH metody, kód API je rozumnější, efektivnější a přehlednější a nedělají se v něm zbytečné kraviny navíc, rozumně se např. už pracuje i s takovými atributy, které lze zaslat na API jako null, ošetření všech možných případů - opravuje \#52

    kompletní validace tel. čísla na backendu i frontendu, vylepšení způsobu validace a dělení čísla na mezery už při psaní o formuláře, následné odstranění mezer až na backendu
    oprava chyby způsobující chybu při úpravě stavu účasti předplacené lekce
* odstranění křížku pro reset react-selectu, aby nedocházelo k vymazání "omylem", když funkcionalita stejně není třeba
* občas se využívají skupinové lekce bez definovaných účastníků (ještě nejsou známí), taková lekce se ale ukazuje jako zrušená a u skupiny chybí info, že nejsou žádní účastníci - potřeba vylepšit zobrazení


\section{Zavedení nástrojů pro usnadnění vývoje a údržby}
podrobny popis jak probihalo zavedeni, co mi to umoznuje, treba i screenshoty, k cemu to bylo pouzito, jak se to osvedcilo - nastroje vychazi ze zvolenych nastroju z reserse
 
\begin{itemize}
\item ...
\item code formatting - prettier, black
\item python zavedeni vulture pro dead code
\item napojeni na Sentry, Slack, logentries, GA
\item zavedeni LGTM, opravy nalezenych problemu
\item travis: zavedeni cache pro yarn a pipenv, zjednoduseni prace s .npmrc, na heroku se neprovadi build a collectstatic
\item zmena react toolkitu (nejdrive nwb, pak test neutrinojs a nakonec custom webpack + porovnani size bundle), viz https://github.com/rodlukas/UP-admin/issues/67 a https://github.com/rodlukas/UP-admin/issues/65
\item ...
\end{itemize}


\chapter{Testování}
implementace samotneho testovani v behave, Selenium
\begin{itemize}
\item popis, jak jsem konkretne tvoril vsechny UI/API testy v behave a seleniu, co bylo za problemy (nacitani, viz heuristika), jak to funguje, jak jsem vsechno udelal
\item coverage 86 \%
\item popsat co vsechno se testuje, ze je to v jazyku gherkin
\item popsat co se netestuje
\item pripadne zminit ze bylo reseno v ramci MI-PYT
\item popsat dalsi problemy s nehezkym API selenia, ktere je pro ruzne jazyky nejednoznacne a nesjednocene (mozna kvuli verzi 3, v4 uz asi bude lepsi...), hrozna dokumentace, ale zase to pouzivaj vsichni tak se vsechno vsude najde
\item pripadne dodat akceptacni testovani
\item niels. heur. analyza, ktera pomohla pak v implementaci prakticky zprovoznit UI testovani, protoze co nevi clovek, nevi ani selenium a tezko se neco testuje (nevi ze se nacita kdyz to nevidi ani clovek, nevi ze se ma cekat..)
\item mozna zminit usability testovani - zejmena proto, ze prakticky na tom stoji dalsi vylepseni v analyze, kde jsem pozoroval lektorku pri bezne praci a na zaklade toho jsme resili, co by chtela zmenit/pridat/upravit; tady neni treba testovat pro jine uzivatele, je to zamerene na interni pouziti 1 clovekem, o to lepsi to ale musi byt:)
\end{itemize}


\chapter{Nasazení}
popis toho jak jsem udelal novej zpusob nasazovani
* django-environ

\section{Zavedení více prostředí}
\begin{itemize}
\item vyvoj BP probihal na lokalu a pak probehl push na GH, travis provedl build a testy a nasadilo se na produkci, to znamená že jsem samozřejmě produkci mohl zbořit (a taky že párkrát zbořil..)
\item jak funguje stage, testing, produkce, demo co kam kdy jde v souvislosti s releasy, k cemu to je, jak se to osvedcilo - dohledavani problemu na stage kdyz se neco stane na produkci
\item (zustava dev a produkce)
\item pokročilé debugování na lokálním i vzdáleném prostředí díky Django Debug Toolbar 
\item vyhoda testing a demo env je ze tam muze kdokoliv, neuvidi nic duverneho (ani pristup do banky tu neni povolen), takze do toho muze vedouci prace, oponent i bezny uzivatel, da se testovat vse na obdobnem prostredi jako pak bude staging, produkce (heroku)
\item + samozrejme je zde moznost spustit na lokalu, ale todle je mnohem rychlejsi (viz GH)
\end{itemize}

\section{Další úpravy}
\begin{itemize}
\item automaticke zalohovani DB
\item ...
\end{itemize}



\chapter{Možná rozšíření}
klasika - co by se dalo delat dal, o co by se appka mohla rozsirit, dalsi technologie, fce

\chapter{Zveřejnění jako open-source}
repo na GH uz je pripravene: https://github.com/rodlukas/UP-admin - vcetne instrukci pro spusteni
\begin{itemize}
\item popsat proc zverejnit - moznost nahlednout na realnou aplikaci s nejnovejsimi technologiemi, jak je nakonfigurovana, inspirovat se + ma reference
\item MIT licence
\item popsat co bylo treba pro zverejneni udelat - priprava vsech casti, vyreseni tokenu, promennych prostredi... + problem: automaticky zverejneni buildu frontendu, aby ho clovek mohl pouzit pri spusteni u sebe na lokalu - protoze se pouziva placena knihovna, ke ktere mam pristup jen ja, takze si to nikdo jiny u sebe nezbuildi (resi to travis a nahraje build do assetu k release) - řešení: na travisu se provede build frontendu a ten se automaticky jako *zip* soubor nahraje k příslušnému github release 
\item na GH  je popsání vlastností, požadavků, postupu instalace a spuštění, testování + připravená vzorová data pro DB (včetně návodu na jejich vložení)
\item (? promenne prostredi mozna zminim i jinde)
\end{itemize}



