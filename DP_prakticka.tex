\chapter{Sběr požadavků a analýza}\label{chap:sberpozadavkuaanalyza}
SI klasika - co všechno je potřeba v appce zlepšit, co chce lektorka, finální přehled funkčních a nefunkčních požadavků

\section{Požadavky}
\begin{itemize}
\item sem bych dal i seznam toho, co bylo na konci BP navrhováno že by se dalo udělat a co z toho se nakonec uděla a neuděla a proč - toto už bude zmineno v teoretické části (v aktuálním řešení), kde uvedu, že pak v analýze vyhodnotím co je potřeba a co ne
\item rozdělil bych to na dvě části - požadavky lektorky a pak zbytek požadavků doporučených z mé strany, které tak nějak vyplynuly z podstaty věci (jako např. dokumentace, refaktoring a mnoho dalšího, bez čeho se nedá moc hnout dál a rozšiřovat)
\end{itemize}

prehled hlavnich pozadavku lektorky (neco muze chybet):
\begin{itemize}
\item js zvyrazneni neaktivity skupin a klientu napric aplikaci https://github.com/rodlukas/UP-admin/issues/85
\item dalsi supr featury https://github.com/rodlukas/UP-admin/issues/73
\item JS Přidat vyhledávání https://github.com/rodlukas/UP-admin/issues/3
\item !! JS efektivnější práce v aplikaci https://github.com/rodlukas/UP-admin/issues/37
\item !! js automatická změna účastníků lekce při úpravě členů skupiny https://github.com/rodlukas/UP-admin/issues/86
\item api+js moznost dat kurzu barvu, zobrazeni barvy napric aplikaci + redesign vsech stranek pro lepsi konzistenci, srozumitelnost a pochopeni (predelany diar, zajemci, lekce v karte...)
\item js přidání nového klienta přímo při přidávání skupiny nebo zájemce
\item zajemci o kurzy
\item napojeni na banku (+cache), 
\item pokročilejší evidence předplacených kurzů (skupiny - citac, indiv - vyber poctu), 
\item kontrola konfliktu lekcí (čas datum)
\item pokročilá validace duplicit apod.
\item automatický odhad kurzu pro nově přidávané lekce,
\item automatické přidání předplacené lekce při omluvě/zrušení lekce ze strany lektorky,
\item kurzy mají nastavitelnou dobu trvání
\item zavedení aktivních a neaktivních klientů a skupin
\item kompletně přepracovaný formulář pro lekce
\item ...
\end{itemize}


zbyle pozadavky (neco muze chybet, nebo patri k pozadavkum lektorky):

\begin{itemize}
\item analyza pruchodu aplikaci (google analytics)
\item frontend odolny proti padu diky ErrorBoundary
\item Pro zrychlení načítání celé aplikace se používá lazy loading React.lazy + React Suspense
\item chybejici dokumentace v kodu -> zavedeni dokumentace v kodu, dokumentace API (swagger), refaktoring
\item mene pozadavku - react context api (attendancestates + nově také viditelné kurzy, aktivní klienti/skupiny - vše napříč aplikací) -> TÍM PÁDEM POTŘEBA PŘEJÍT NA NOVÝ REACT -> je tedy potřeba provést i migraci na nový lifecycle reactu
\item pomalé SQL dotazy- obri optimalizace dotazu na DB (>4x zrychleni) diky DJDT, pokrocile optimalizace
\item osetreni vstupu (vcetne konecne plne funkcniho API)
\item všude načítání, když jsou nějaké požadavky
\item optimalizace frontendu - odstraneni komponent definovanych primo v render - mohlo zpusobit potize s vykonem i nechtenym prekreslovanim vnorenych komponent
\item js vylepseni notifikaci - notifikace doplnene o nadpis a ikonu, sjednoceny font, komponenta pro obsah notifikace, delsi doba zobrazeni error notifikace
\item js pozdrzeni requestu DashboardDay v diari pri preklikavani tydnu 
\item api krestni jmeno klienta reprezentovane "firstname"
\item zavedeni referrer-policy (same-origin), bezpecnejsi konfigurace lokalniho prostredi
\item oprava zabezpeceni aplikace + automaticke presmerovani na https + HSTS
\item opravy nevalidniho html a css (diky tomu lepsi zarovnani tlacitek)
\item nova funkce: datum pridani zajemce o kurz
\item template strings js
\item opravy nekonzistentnich stavu
\item oprava doporucovani kurzu, doporuceni data a casu (+ 7 dni)
\item js plynule posouvani v modalnim okne v Safari (iOS)
\item vylepseni responzivity
\item kazda stranka ma svuj title
\item pridani lazy loadingu a react suspense - umozni kratsi prvni nacteni appky, zalozene na react-routeru
\item klikani na labely vsude funguje
\item zavedeni prvnich hooku, drobna vylepseni komponent - prechod na funkcionalni komponenty, upravy komponent (napr. zbytecne pouzivaly stav nebo se zbytecne casto updatovaly, ikdyz to nebylo potreba) + **odstranění zbytečného stavu odstranilo i chybu, která byla už od počátku v aplikaci a způsobovala občasné nefungování komponenty s aktuálním stavem účasti - zobrazil se jiný stav, než byl skutečný, v závislosti na pořadí obdržení odpovědí od serveru, to se začalo víc projevovat při nedávných změnách architektury reactu**
\item podrobnejsi chyby z api
\item osetreni v api (IO...)
\item mnoho promennych prostredi
\item  procházení diářem šipkami na klávesnici + request až po sekundě
\item js react-select escape zavře i modal okno https://github.com/rodlukas/UP-admin/issues/84
\item js komponenta pro zobrazeni lekci https://github.com/rodlukas/UP-admin/issues/71
\item chyby při práci s modálními okny https://github.com/rodlukas/UP-admin/issues/95 - issue reactstrapu, vyjde nova verze
\item react fontawesome PR pro vylepseni typescript podpory (ID atribut..)
\item js upozornění na ztrátu dat formuláře https://github.com/rodlukas/UP-admin/issues/83
\item DRF-JWT problémy, dotazy na SQL DB ikdyž nemají být - https://github.com/rodlukas/UP-admin/issues/51 - přechod na jinou knihovnu a zde PR na překlad do CZ
\item python refaktoring https://github.com/rodlukas/UP-admin/issues/93
\item js useEffect - exhaustive deps https://github.com/rodlukas/UP-admin/issues/96
\item drf zpracovavat ProtectedError https://github.com/rodlukas/UP-admin/issues/70
\item zvýšení usability/acessibility https://github.com/rodlukas/UP-admin/issues/27
\item reseni Error: Loading chunk 11 failed. https://github.com/rodlukas/UP-admin/issues/92
\item ...
\end{itemize}

\section{Procesy a entity}
je treba zjistit a uvest informace k novym entitam a procesum, napr. jak funguji zajemci o kurz, jak evidovani predplacenych lekci.. aby se na zaklade toho mohl udelat navrh


\chapter{Návrh}
navrh upravene DB a API a pripadne dalsich veci

\chapter{Implementace}
součástí sekce bude i sypani popela na hlavu za chyby v bakalarce - popisu co vsechno se delo, co bylo za problemy (napr bad patterny v reactu, coz zpusobilo ruzne problemy)

\section{Implementace požadavků}
IMPLEMENTACE POZADAVKU z analyzy vyse

\section{Zavedení nástrojů pro usnadnění vývoje a údržby}
podrobny popis jak probihalo zavedeni, co mi to umoznuje, treba i screenshoty, k cemu to bylo pouzito, jak se to osvedcilo - nastroje vychazi ze zvolenych nastroju z reserse
 
\begin{itemize}
\item ...
\item code formatting - prettier, black
\item python zavedeni vulture pro dead code
\item napojeni na Sentry, Slack, logentries, GA
\item zavedeni LGTM, opravy nalezenych problemu
\item travis: zavedeni cache pro yarn a pipenv, zjednoduseni prace s .npmrc, na heroku se neprovadi build a collectstatic
\item zmena react toolkitu (nejdrive nwb, pak test neutrinojs a nakonec custom webpack + porovnani size bundle), viz https://github.com/rodlukas/UP-admin/issues/67 a https://github.com/rodlukas/UP-admin/issues/65
\item ...
\end{itemize}


\chapter{Testování}
implementace samotneho testovani v behave, Selenium
\begin{itemize}
\item popis, jak jsem konkretne tvoril vsechny UI/API testy v behave a seleniu, co bylo za problemy (nacitani, viz heuristika), jak to funguje, jak jsem vsechno udelal
\item coverage 86 \%
\item popsat co vsechno se testuje, ze je to v jazyku gherkin
\item popsat co se netestuje
\item pripadne zminit ze bylo reseno v ramci MI-PYT
\item popsat dalsi problemy s nehezkym API selenia, ktere je pro ruzne jazyky nejednoznacne a nesjednocene (mozna kvuli verzi 3, v4 uz asi bude lepsi...), hrozna dokumentace, ale zase to pouzivaj vsichni tak se vsechno vsude najde
\item pripadne dodat akceptacni testovani
\item niels. heur. analyza, ktera pomohla pak v implementaci prakticky zprovoznit UI testovani, protoze co nevi clovek, nevi ani selenium a tezko se neco testuje (nevi ze se nacita kdyz to nevidi ani clovek, nevi ze se ma cekat..)
\item mozna zminit usability testovani - zejmena proto, ze prakticky na tom stoji dalsi vylepseni v analyze, kde jsem pozoroval lektorku pri bezne praci a na zaklade toho jsme resili, co by chtela zmenit/pridat/upravit; tady neni treba testovat pro jine uzivatele, je to zamerene na interni pouziti 1 clovekem, o to lepsi to ale musi byt:)
\end{itemize}


\chapter{Nasazení}
popis toho jak jsem udelal novej zpusob nasazovani

\section{Zavedení více prostředí}
\begin{itemize}
\item vyvoj BP probihal na lokalu a pak probehl push na GH, travis provedl build a testy a nasadilo se na produkci, to znamená že jsem samozřejmě produkci mohl zbořit (a taky že párkrát zbořil..)
\item jak funguje stage, testing, produkce, demo co kam kdy jde v souvislosti s releasy, k cemu to je, jak se to osvedcilo - dohledavani problemu na stage kdyz se neco stane na produkci
\item (zustava dev a produkce)
\item pokročilé debugování na lokálním i vzdáleném prostředí díky Django Debug Toolbar 
\item vyhoda testing a demo env je ze tam muze kdokoliv, neuvidi nic duverneho (ani pristup do banky tu neni povolen), takze do toho muze vedouci prace, oponent i bezny uzivatel, da se testovat vse na obdobnem prostredi jako pak bude staging, produkce (heroku)
\item + samozrejme je zde moznost spustit na lokalu, ale todle je mnohem rychlejsi (viz GH)
\end{itemize}

\section{Další úpravy}
\begin{itemize}
\item automaticke zalohovani DB
\item ...
\end{itemize}



\chapter{Možná rozšíření}
klasika - co by se dalo delat dal, o co by se appka mohla rozsirit, dalsi technologie, fce

\chapter{Zveřejnění jako open-source}
repo na GH uz je pripravene: https://github.com/rodlukas/UP-admin - vcetne instrukci pro spusteni
\begin{itemize}
\item popsat proc zverejnit - moznost nahlednout na realnou aplikaci s nejnovejsimi technologiemi, jak je nakonfigurovana, inspirovat se + ma reference
\item MIT licence
\item popsat co bylo treba pro zverejneni udelat - priprava vsech casti, vyreseni tokenu, promennych prostredi... + problem: automaticky zverejneni buildu frontendu, aby ho clovek mohl pouzit pri spusteni u sebe na lokalu - protoze se pouziva placena knihovna, ke ktere mam pristup jen ja, takze si to nikdo jiny u sebe nezbuildi (resi to travis a nahraje build do assetu k release) - řešení: na travisu se provede build frontendu a ten se automaticky jako *zip* soubor nahraje k příslušnému github release 
\item na GH  je popsání vlastností, požadavků, postupu instalace a spuštění, testování + připravená vzorová data pro DB (včetně návodu na jejich vložení)
\item (? promenne prostredi mozna zminim i jinde)
\end{itemize}



