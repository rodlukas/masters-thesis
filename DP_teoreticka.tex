\chapter{Aktuální řešení}
% todo citace, odkazy, lepsi popisy technologii (napr webpack)
V této kapitole stručně popíši verzi aplikace, která byla vytvořena v rámci bakalářské práce \cite{bp} a kterou budu v praktické části rozšiřovat. Popíši implementované funkční i nefunkční požadavky, použité technologie a také konfiguraci prostředí a způsob nasazování. Na závěr stručně shrnu, jaké možné plány na rozšíření byly v rámci bakalářské práce nastíněny.

\section{Implementované požadavky}

\subsection{Implementované funkční požadavky}

Nejprve shrnu implementované funkční požadavky \cite{bp}:
%TODO odkaz na obrazek puvodniho datoveho modelu
\begin{itemize}
    \item \textbf{evidence klientů:} evidování základních informací o klientovi,
    \item \textbf{evidence lekcí klientů:} evidování základních informací o lekcích (včetně stavů účasti všech účastníků a stavu jejich platby za danou lekci), lekce mohou být pro jednotlivce nebo pro skupiny, každá náleží nějakému kurzu,
    \item \textbf{evidence předplacených lekcí:} předplacená lekce je řešená jako lekce, která nemá datum a čas konání,
    \item \textbf{evidence kurzů a stavů účasti:} pro použití při evidenci lekcí,
    \item \textbf{přehled lekcí pro aktuální den:} zobrazení pro dnešní lekce (kromě zrušených),
    \item \textbf{karta klienta a skupiny}: zobrazení všech informací o klientovi/skupině včetně všech lekcí,
    \item \textbf{upozornění na platbu příště:} když už klient nemá žádné předplacené lekce, zobrazí se u poslední placené lekce upozornění na fakt, že má příště zaplatit,
    \item \textbf{pořadové číslo lekce:} u lekce se zobrazí (automaticky vypočítáno), o kolikátou navštívenou lekci v pořadí se jedná,
    \item \textbf{týdenní přehled:} zobrazení lekcí jako v diáři.
\end{itemize}

\subsection{Implementované nefunkční požadavky}

Následuje shrnutí nefunkčních požadavků \cite{bp}:
\begin{itemize}
    \item \textbf{kompatibilita s webovými prohlížeči:} aplikace je plně funkční a kompatibilní s běžnými webovými prohlížeči (Google Chrome, Mozilla Firefox, Microsoft Edge, Apple Safari) v posledních verzích, důraz ja především kladen na desktopový prohlížeč Mozilla Firefox, na kterém se aplikace používá primárně,
    \item \textbf{podpora široké škály zařízeních:} aplikace je responzivní a korektně se zobrazuje na všech zařízeních používaných lektorkou -- iPad s iOS~13.3 a 9,7palcovým displejem, Nokia~5 s Androidem~9.0 a 5,2palcovým~displejem, notebook s Windows~10 s rozlišením 1920~×~1080 a 15,6palcovým~displejem,
    \item \textbf{připravenost na rozšíření a údržbu}: kód byl tvořen s důrazem na budoucí možná rozšíření (např. použití konstant, respektování principu DRY (\enquote{Don\textquotesingle t repeat yourself}) ad.),
    \item \textbf{bezpečnost:} JWT (JSON Web Token) autentizace a další produkční konfigurace pro zabezpečení aplikace,
    \item \textbf{srozumitelné a jednoduché rozhraní aplikace:} iterativní návrh a implementace UI za neustálé spolupráce s lektorkou pro dosažení co nejlepší použitelnosti.
\end{itemize}


\section{Použité technologie}
\subsection{Serverová část}

Serverová část aplikace \cite{bp} je napsána v Pythonu~3.6.5 s webovým frameworkem \href{https://www.djangoproject.com/}{Django~2.0.5}. Pro správu závislostí se používá pouze jednoduchý soubor \verb|requirements.txt| obsahující specifikace přesných verzí knihoven (bez povolení jakkoliv malých aktualizací). 

Aplikace vystavuje REST~API postavené na frameworku \href{https://www.django-rest-framework.org/}{Django~REST~framework~3.8.2}. Na produkci se používá webový server \href{http://gunicorn.org/}{Gunicorn~19.8.1} spolu s knihovnou \href{http://whitenoise.evans.io/en/stable/}{WhiteNoise~3.3.1} pro efektivní servírování zkomprimovaných statických souborů \cite{whitenoise}.

\subsection{Klientská část}

Klientská část aplikace \cite{bp} je napsána v JS (JavaScript) ve standardu ECMAScript®~2018. Pro správu závislostí se používá soubor \verb|package.json|, v němž jsou verze přibližně poloviny knihoven definovány napevno (bez možnosti jakkoliv malé aktualizace), druhá polovina knihoven přijímá malé aktualizace. 

Klientskou část aplikace lze klasifikovat jako: 
\begin{itemize}
    \item \textbf{SPA} (Single-Page Application), tedy aplikaci běžící přímo u klienta v prohlížeči nevyžadující znovunačítání při přecházení mezi stránkami \cite{spa1} a
    \item \textbf{CSR} (Client-Side Rendering), tedy aplikaci, která je klientovi doručena jako jednoduchý HTML (Hypertext Markup Language) soubor s odkazy na JS/CSS (Cascading Style Sheets) soubory \cite{csr-ssr}.
\end{itemize}

Je postavena na knihovně \href{https://reactjs.org/}{React~16.3} spolu s UI frameworkem \href{https://getbootstrap.com}{Bootstrap~4.1} a související knihovnou \href{https://reactstrap.github.io/}{Reactstrap~5.0}, která umožňuje jednoduché použití Bootstrap komponent v Reactu \cite{reactstrap}. Pro asynchronní požadavky na REST API využívá knihovnu \href{https://github.com/axios/axios}{axios~0.18}. 

Konfigurace celé klientské části stojí na nástroji \href{https://github.com/facebook/create-react-app}{create-react-app~1}, který umožňuje \cite{cra} vytvářet React aplikace bez počáteční konfigurace. Vzhledem k použití Djanga bylo ale potřeba \cite{bp} pomocí příkazu \verb|eject| \enquote{vysunout} celou konfiguraci klientské části a pomocí knihoven \href{https://github.com/owais/webpack-bundle-tracker}{webpack-bundle-tracker} a  \href{https://github.com/owais/django-webpack-loader}{django-webpack-loader} propojit Django s nástrojem Webpack. Webpack zde umožňuje mj. spouštět vývojový server pro klientskou část a vytvářet z jednotlivých modulů klientské části balíčky, které lze pak po zadaných transformacích servírovat na produkci pro běžné webové prohlížeče \cite{webpack-ackee}.

V případě vývoje na lokálním stroji se používá \cite{bp} pro serverovou část vývojový Django server a pro klientskou část \href{https://github.com/webpack/webpack-dev-server}{webpack-dev-server}, oba nabízejí podporu pro \enquote{hot reloading} (tedy okamžité automatické projevení změn v kódu bez kompletního znovunačtení aplikace \cite{webpack-docs-hmr}) a v tomto ohledu bylo vše i díky zmíněným knihovnám zprovozněno.

\section{Prostředí, testování a nasazování}
Aplikace \cite{bp} je verzována v privátním repozitáři na serveru GitHub. Při každém nahrání nové revize na server (\verb|push|) se na integračním serveru \href{https://travis-ci.com/}{Travis~CI} spustí sestavení aplikace, vytvoří se testovací databáze a spustí se základní testy (viz níže). Výsledné pokrytí kódu se poté nahraje na platformu \href{https://codecov.io/}{codecov.io} pro pokročilé statistiky o testování \cite{codecov}. Pokud vše na Travisu proběhne v pořádku, začne nasazení na produkční server běžící na \href{https://www.heroku.com/}{Heroku}. Nasazení na Heroku probíhá tak, že se přímo na něm spouští celé sestavení aplikace znovu, zmigruje se databáze a aplikace se nasadí. I tento průběh lze sledovat přímo z Travis terminálu.

Spouštěné testy jsou základní a velmi jednoduché, otestují \cite{bp}:
\begin{itemize}
    \item přidání klienta a uživatele do databáze přes Django modely,
    \item zda Django uživateli zobrazí správnou stránku při příchodu do aplikace (neřeší, zda se pak vůbec JS aplikace vyrenderuje),
    \item funkčnost API požadavků -- proběhne autorizace (a tedy získání JWT tokenu) a pokus o vytvoření nového klienta přes API.
\end{itemize}

Jak je vidět, testy byly skutečně pouze velmi povrchní, dalo by se říci, že se jedná o smoke testy. Také je třeba zdůraznit, že provedení \verb|push| na repozitář mělo za následek okamžité nasazení na produkci, pokud sestavení a tyto základní testy prošly. To může být nedozírné následky. I proto se v dalších kapitolách této teoretické části budu zaobírat možnostmi zlepšení.

\section{Plán rozšíření z bakalářské práce}
Na závěr bakalářské práce \cite{bp} bylo zmíněno několik možných rozšíření aplikace. Následuje jejich přehledný stručný přehled, v kapitole~\ref{pozadavkyanalyza} se mimo jiné na některá z nich také dostane.
Přehled možných rozšíření \cite{bp}:
\begin{itemize}
    \item \textbf{vylepšení předplacených lekcí:} pohodlnější způsob zaznamenávání předplacených lekcí, pro skupiny je to velmi krkolomné a nepohodlné, pro jednotlivce také,
    \item \textbf{kontrola časového konfliktu lekcí:} aby se dvě nezrušené lekce vzhledem k datumu, času a délce trvání nijak nepřekrývaly,
    \item \textbf{vyhledávání v aplikaci:} např. vyhledávání klientů,
    \item \textbf{evidování zájemců o kurz:} pro plánování nových lekcí kurzů pro jednotlivce a skupiny,
    \item \textbf{evidence pomůcek a učebnic},
    \item \textbf{testy:} doplnění dalších testů a vysoké pokrytí kódu,
    \item \textbf{migrace na nový React:} k vydání verze 16.3 došlo na konci vývoje aplikace v rámci bakalářské práce, např. došlo ke změnám v API (životní cyklus komponent) \cite{react-blog-163},
    \item \textbf{React Context API:} analýza možností využití Context API v rámci nového Reactu \cite{react-blog-163}, zejména by pravděpodobně pomohlo snížit např. počet přístupů do API z klientské části napříč aplikací,
    \item \textbf{offline přístup:} analýza možností řešení offline přístupu, např. automatické ukládání do Google kalendáře, progresivní webové aplikace apod. a s tím související další oblasti jako SSR (Server-Side Rendering) -- tedy klient obdrží od serveru HTML dokument připravený k vyrenderování, oproti CSR, kde by klient pro vyrenderování aplikace musel čekat na stažení a spuštění JS souborů \cite{csr-ssr}.
\end{itemize}

\chapter{Možnosti automatizovaného testování}

\begin{itemize}
\item popsat ruzne možnosti testování - BDD, TDD...
\end{itemize}
\begin{itemize}
\item ruzny scope - API, UI, unit.. vyhody, nevyhody...
\end{itemize}
\begin{itemize}
\item popsat selenium a alternativy (capybara?...)
\end{itemize}


\chapter{Konfigurace více prostředí}
obsah: Deployment environments a s tím související Release management


\begin{itemize}
\item přehled toho, jak se můžou řešit různý prostředí, např. staging, testing, produkce apod.
\end{itemize}
\begin{itemize}
\item přehled toho jak s tím souvisí releasy, co kam může jít a k čemu je to dobrý, jak se to dá různě dělat a souvislosti (např. nemusí to dělat travis na základě releasu, ale třeba na základě podmínky proměnné prostředí nebo branch production apod.)
\end{itemize}

viz:

\begin{itemize}
\item https://en.wikipedia.org/wiki/Deployment\_environment
\item https://docs.gitlab.com/ee/ci/environments.html
\item https://cs.wikipedia.org/wiki/Release\_management
\item https://docs.microsoft.com/cs-cz/aspnet/web-forms/overview/deployment/configuring-server-environments-for-web-deployment/scenario-configuring-a-staging-environment-for-web-deployment
\item http://guides.beanstalkapp.com/deployments/best-practices.html
\item ...
\end{itemize}

\chapter{Nástroje pro usnadnění vývoje a údržby}
rešerše nástrojů pro:
\begin{itemize}
\item odchytavani chyb (sentry...)
\item logovani (logentries...)
\item staticka analyza a kontrola kodu, hledani chyb, zero-days zranitelnosti, hodnoceni kvality kodu (LGTM...)
\item code formattery (prettier, black...) + souvislost s flake8 apod.
\item analyza pruchodu aplikaci (google analytics)
\item pokryti kodu (codecov, coverage.py...)
\item pripadne nejake pokrocile veci z travisu co se pouzivaji
\item lintery (ESlint.. + styly airbnb, standardjs, info o spouste pluginech, standardech, vyvoji ve spolupraci s TS)
\item static type checking pro frontend i backend (python, js) - moznosti, ze python ma builtin, JS nema (uvest proc, viz ecma vyjadreni), jak se to v JS resi a srovnat: FlowJS, Typescript + jak jsou na tom ostatni jako coffeescript, dart, kotlin..; u TS zminit podporu babel
\item + hledani dead code - python vulture
\item nastroje pro vyvoj frontendu - bakalarka byla postavena na CRA (create-react-app), ktera byla ejectnuta (takze update reactu byl prakticky nemozny), z toho duvodu se nejprv preslo na nastroj nwb, ktery ale pozdeji nebyl udrzovany a doslo ke zkusebni migraci na neutrinojs (souvisi s mozillou) a na vlastni reseni pres webpack - tady bych teda prosel, co tyhle nastroje umoznuji, ze vlastne stavi nad webpackem, ze je to vlastne abstraktni vrstva nad nim, ktera vsechno zjednodusuje, ale nevyhoda je, kdyz ten nastroj neco neumoznuje (viz nwb), tak mas smulu.. takze pak nejdou lintery, nejde TS... zminit moznosti nastroju, jake nastroje jsou, vyhody a nevyhody; v ramci implementace pak zminim cely pribeh s nwb, neutrinojs a custom webpackem, na zaver uvedu proc jsem vse zmigroval na custom webpack ikdyz neutrinojs fungovalo taky
\end{itemize}


\chapter{Zvolené technologie}
Na základě rešerše zvolím technologie, které pak zakomponuji v implementaci do appky - oargumentuji volbu..

tzn:

\begin{itemize}
\item 1. cim testovat a jak
\item 2. jaka prostredi a jak delat releasy, jak to vse bude fungovat (a fakt funguje)
\item 3. zvolene nastroje pro usnadneni vyvoje a udrzby
\end{itemize}