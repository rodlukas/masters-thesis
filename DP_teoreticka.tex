\chapter{Aktuální řešení}
% todo citace, odkazy, lepsi popisy technologii (napr webpack)
V této kapitole stručně popíši verzi aplikace, která byla vytvořena v rámci bakalářské práce \cite{bp} a kterou budu v praktické části rozšiřovat. Popíši implementované funkční i nefunkční požadavky, použité technologie a také konfiguraci prostředí a způsob nasazování. Na závěr stručně shrnu, jaké možné plány na rozšíření byly v rámci bakalářské práce nastíněny.

\section{Implementované požadavky}

\subsection{Implementované funkční požadavky}

Nejprve shrnu implementované funkční požadavky \cite{bp}:
%TODO odkaz na obrazek puvodniho datoveho modelu
\begin{itemize}
    \item \textbf{evidence klientů:} evidování základních informací o klientovi,
    \item \textbf{evidence lekcí klientů:} evidování základních informací o lekcích (včetně stavů účasti všech účastníků a stavu jejich platby za danou lekci), lekce mohou být pro jednotlivce nebo pro skupiny, každá náleží nějakému kurzu,
    \item \textbf{evidence předplacených lekcí:} předplacená lekce je řešená jako lekce, která nemá datum a čas konání,
    \item \textbf{evidence kurzů a stavů účasti:} pro použití při evidenci lekcí,
    \item \textbf{přehled lekcí pro aktuální den:} zobrazení pro dnešní lekce (kromě zrušených),
    \item \textbf{karta klienta a skupiny}: zobrazení všech informací o klientovi/skupině včetně všech lekcí,
    \item \textbf{upozornění na platbu příště:} když už klient nemá žádné předplacené lekce, zobrazí se u poslední placené lekce upozornění na fakt, že má příště zaplatit,
    \item \textbf{pořadové číslo lekce:} u lekce se zobrazí (automaticky vypočítáno), o kolikátou navštívenou lekci v pořadí se jedná,
    \item \textbf{týdenní přehled:} zobrazení lekcí jako v diáři.
\end{itemize}

\subsection{Implementované nefunkční požadavky}

Následuje shrnutí nefunkčních požadavků \cite{bp}:
\begin{itemize}
    \item \textbf{kompatibilita s webovými prohlížeči:} aplikace je plně funkční a kompatibilní s běžnými webovými prohlížeči (Google Chrome, Mozilla Firefox, Microsoft Edge, Apple Safari) v posledních verzích, důraz ja především kladen na desktopový prohlížeč Mozilla Firefox, na kterém se aplikace používá primárně,
    \item \textbf{podpora široké škály zařízeních:} aplikace je responzivní a korektně se zobrazuje na všech zařízeních používaných lektorkou -- iPad s iOS~13.3 a 9,7palcovým displejem, Nokia~5 s Androidem~9.0 a 5,2palcovým~displejem, notebook s Windows~10 s rozlišením 1920~×~1080 a 15,6palcovým~displejem,
    \item \textbf{připravenost na rozšíření a údržbu}: kód byl tvořen s důrazem na budoucí možná rozšíření (např. použití konstant, respektování principu DRY (\enquote{Don\textquotesingle t repeat yourself}) ad.),
    \item \textbf{bezpečnost:} JWT (JSON Web Token) autentizace a další produkční konfigurace pro zabezpečení aplikace,
    \item \textbf{srozumitelné a jednoduché rozhraní aplikace:} iterativní návrh a implementace UI za neustálé spolupráce s lektorkou pro dosažení co nejlepší použitelnosti.
\end{itemize}


\section{Použité technologie}
\subsection{Serverová část}

Serverová část aplikace \cite{bp} je napsána v Pythonu~3.6.5 s webovým frameworkem \href{https://www.djangoproject.com/}{Django~2.0.5}. Pro správu závislostí se používá pouze jednoduchý soubor \verb|requirements.txt| obsahující specifikace přesných verzí knihoven (bez povolení jakkoliv malých aktualizací). 

Aplikace vystavuje REST~API postavené na frameworku \href{https://www.django-rest-framework.org/}{Django~REST~framework~3.8.2}. Na produkci se používá webový server \href{http://gunicorn.org/}{Gunicorn~19.8.1} spolu s knihovnou \href{http://whitenoise.evans.io/en/stable/}{WhiteNoise~3.3.1} pro efektivní servírování zkomprimovaných statických souborů \cite{whitenoise}.

\subsection{Klientská část}

Klientská část aplikace \cite{bp} je napsána v JS (JavaScript) ve standardu ECMAScript®~2018. Pro správu závislostí se používá soubor \verb|package.json|, v němž jsou verze přibližně poloviny knihoven definovány napevno (bez možnosti jakkoliv malé aktualizace), druhá polovina knihoven přijímá malé aktualizace. 

Klientskou část aplikace lze klasifikovat jako: 
\begin{itemize}
    \item \textbf{SPA} (Single-Page Application), tedy aplikaci běžící přímo u klienta v prohlížeči nevyžadující znovunačítání při přecházení mezi stránkami \cite{spa1} a
    \item \textbf{CSR} (Client-Side Rendering), tedy aplikaci, která je klientovi doručena jako jednoduchý HTML (Hypertext Markup Language) soubor s odkazy na JS/CSS (Cascading Style Sheets) soubory \cite{csr-ssr}.
\end{itemize}

Je postavena na knihovně \href{https://reactjs.org/}{React~16.3} spolu s UI frameworkem \href{https://getbootstrap.com}{Bootstrap~4.1} a související knihovnou \href{https://reactstrap.github.io/}{Reactstrap~5.0}, která umožňuje jednoduché použití Bootstrap komponent v Reactu \cite{reactstrap}. Pro asynchronní požadavky na REST API využívá knihovnu \href{https://github.com/axios/axios}{axios~0.18}. 

Konfigurace celé klientské části stojí na nástroji \href{https://github.com/facebook/create-react-app}{create-react-app~1}, který umožňuje \cite{cra} vytvářet React aplikace bez počáteční konfigurace. Vzhledem k použití Djanga bylo ale potřeba \cite{bp} pomocí příkazu \verb|eject| \enquote{vysunout} celou konfiguraci klientské části a pomocí knihoven \href{https://github.com/owais/webpack-bundle-tracker}{webpack-bundle-tracker} a  \href{https://github.com/owais/django-webpack-loader}{django-webpack-loader} propojit Django s nástrojem Webpack. Webpack zde umožňuje mj. spouštět vývojový server pro klientskou část a vytvářet z jednotlivých modulů klientské části balíčky, které lze pak po zadaných transformacích servírovat na produkci pro běžné webové prohlížeče \cite{webpack-ackee}.

V případě vývoje na lokálním stroji se používá \cite{bp} pro serverovou část vývojový Django server a pro klientskou část \href{https://github.com/webpack/webpack-dev-server}{webpack-dev-server}, oba nabízejí podporu pro \enquote{hot reloading} (tedy okamžité automatické projevení změn v kódu bez kompletního znovunačtení aplikace \cite{webpack-docs-hmr}) a v tomto ohledu bylo vše i díky zmíněným knihovnám zprovozněno.

\section{Prostředí, testování a nasazování}
Aplikace \cite{bp} je verzována v privátním repozitáři na serveru GitHub. Při každém nahrání nové revize na server (\verb|push|) se na integračním serveru \href{https://travis-ci.com/}{Travis~CI} spustí sestavení aplikace, vytvoří se testovací databáze a spustí se základní testy (viz níže). Výsledné pokrytí kódu se poté nahraje na platformu \href{https://codecov.io/}{codecov.io} pro pokročilé statistiky o testování \cite{codecov}. Pokud vše na Travisu proběhne v pořádku, začne nasazení na produkční server běžící na \href{https://www.heroku.com/}{Heroku}. Nasazení na Heroku probíhá tak, že se přímo na něm spouští celé sestavení aplikace znovu, zmigruje se databáze a aplikace se nasadí. I tento průběh lze sledovat přímo z Travis terminálu.

Spouštěné testy jsou základní a velmi jednoduché, otestují \cite{bp}:
\begin{itemize}
    \item přidání klienta a uživatele do databáze přes Django modely,
    \item zda Django uživateli zobrazí správnou stránku při příchodu do aplikace (neřeší, zda se pak vůbec JS aplikace vyrenderuje),
    \item funkčnost API požadavků -- proběhne autorizace (a tedy získání JWT tokenu) a pokus o vytvoření nového klienta přes API.
\end{itemize}

Jak je vidět, testy byly skutečně pouze velmi povrchní, dalo by se říci, že se jedná o smoke testy. Také je třeba zdůraznit, že provedení \verb|push| na repozitář mělo za následek okamžité nasazení na produkci, pokud sestavení a tyto základní testy prošly. To může být nedozírné následky. I proto se v dalších kapitolách této teoretické části budu zaobírat možnostmi zlepšení.

\section{Plán rozšíření z bakalářské práce}
Na závěr bakalářské práce \cite{bp} bylo zmíněno několik možných rozšíření aplikace. Následuje jejich přehledný stručný přehled, v kapitole~\ref{pozadavkyanalyza} se mimo jiné na některá z nich také dostane.
Přehled možných rozšíření \cite{bp}:
\begin{itemize}
    \item \textbf{vylepšení předplacených lekcí:} pohodlnější způsob zaznamenávání předplacených lekcí, pro skupiny je to velmi krkolomné a nepohodlné, pro jednotlivce také,
    \item \textbf{kontrola časového konfliktu lekcí:} aby se dvě nezrušené lekce vzhledem k datumu, času a délce trvání nijak nepřekrývaly,
    \item \textbf{vyhledávání v aplikaci:} např. vyhledávání klientů,
    \item \textbf{evidování zájemců o kurz:} pro plánování nových lekcí kurzů pro jednotlivce a skupiny,
    \item \textbf{evidence pomůcek a učebnic},
    \item \textbf{testy:} doplnění dalších testů a vysoké pokrytí kódu,
    \item \textbf{migrace na nový React:} k vydání verze 16.3 došlo na konci vývoje aplikace v rámci bakalářské práce, např. došlo ke změnám v API (životní cyklus komponent) \cite{react-blog-163},
    \item \textbf{React Context API:} analýza možností využití Context API v rámci nového Reactu \cite{react-blog-163}, zejména by pravděpodobně pomohlo snížit např. počet přístupů do API z klientské části napříč aplikací,
    \item \textbf{offline přístup:} analýza možností řešení offline přístupu, např. automatické ukládání do Google kalendáře, progresivní webové aplikace apod. a s tím související další oblasti jako SSR (Server-Side Rendering) -- tedy klient obdrží od serveru HTML dokument připravený k vyrenderování, oproti CSR, kde by klient pro vyrenderování aplikace musel čekat na stažení a spuštění JS souborů \cite{csr-ssr}.
\end{itemize}

\chapter{Možnosti automatizovaného testování}

Testování je důležitou součástí softwarového vývoje \cite{test-bdo}. V této kapitole nejprve srovnám manuální a automatizované testování, poté se zaměřím na strukturu samotných testů -- jaká jsou doporučení ohledně zaměření testů a kterých by mělo být nejvíce, resp. nejméně. Některé metodiky softwarového vývoje jsou zaměřené na způsob testování, proto uvedu hlavní informace o těchto metodikách a také s nimi související nástroje, které umožňují tyto metodiky zavést. Uvedu i další nástroje pro testování UI.

\section{Srovnání manuálního a automatizovaného testování}
Dříve, když trval vývojový cyklus několik měsíců až let, se obvykle testovalo převážně pouze manuálně \cite{test-kitner}. Tedy na základě scénářů testeři prováděli testy \cite{test-bdo}. Dnes, v době agilního vývoje a rychlých dodávek, je potřeba větší část testování provádět automatizovaně \cite{test-kitner}. Tedy naprogramovat testy a poté je automaticky spouštět testovacím nástrojem \cite{test-bdo}. Díky tomu se týmy mohou dozvědět o chybě v řádu sekund a minut místo dnů a týdnů \cite{test-fowler}.

\textbf{Výhodou} automatizovaného testování je:
\begin{itemize}
    \item šetří čas, je rychlejší, umožní rychleji vydávat nové verze (kompletní manuální testování bylo úzkým hrdlem v životním cyklu vývoje softwaru) \cite{test-kitner, test-genez, test-cd},
    \item odhalí chyby v dřívějších fázích, tedy snižuje náklady \cite{test-kitner},
    \item dělá testování \enquote{zábavnější} a umožní efektivnější manuální testování -- odstraňuje běžnou neustále opakující se rutinu při manuálním testování \cite{test-kitner, test-perfecto},
    \item dochází ke zpřesnění testů a vyšší spolehlivosti, protože tester se při neustálém opakování na všech operačních systémech a prohlížečích může splést nebo něco přehlédnout \cite{test-kitner, test-genez},
    \item možnost vyššího pokrytí kódu testy a odhalení více chyb díky jednoduchému zavedení více permutací různých zařízení a operačních systémů \cite{test-perfecto},
    \item umožňuje zavést průběžnou integraci a dodávání \cite{test-kitner2}.
\end{itemize}

\textbf{Nevýhodou} automatizovaných testů je:
\begin{itemize}
    \item při nesprávném rozhodnutí o oblasti, kterou budeme optimalizovat, můžeme stovky hodin strávit při vytváření testů, které nakonec nebudou vůbec nacházet podstatné chyby \cite{test-kitner},
    \item nevhodně napsané testy mohou dát falešnou naději, že vše funguje, ačkoliv se v aplikaci vyskytují závažné chyby \cite{test-devqa},
    \item nikdy zcela nenahradí lidské pozorování (při manuálním testování) a nemohou garantovat přívětivost k uživateli či pozitivní uživatelský prožitek \cite{test-genez},
    \item vyšší pravděpodobnost falešně pozitivních výsledků, následná analýza problému je náročná, protože je třeba zjistit, zda se jedná o chybu aplikace či testu \cite{test-perfecto},
    \item nutnost neustálé údržby testů -- při zavedení změny v aplikaci je třeba vše co nejdříve (ideálně okamžitě) projevit do testů \cite{test-swsrovnani}.
\end{itemize}

Jak je vidět ze shrnutí výhod a nevýhod automatizovaného testování, manuální testování má v softwarovém vývoji stále své místo. Zejména kvůli lidskému faktoru a v případech, kdy se konkrétní test nemá spouštět opakovaně kvůli časové náročnosti vytvoření automatizovaného testu \cite{test-bdo}. Na možnosti aplikace manuálního a automatizovaného testování lze také nahlížet z hlediska různých testovacích cyklů softwaru \cite{test-bdo}, tedy manuální testování je vhodné na počátku vývoje/iterace a na závěr při testech použitelnosti, kde je důležitý lidský faktor. Automatizované testování je pak vhodné pro fázi testování výkonu a také pro fázi regresních testů \cite{test-bdo}, ty se využívají při opětovném testování stávajících funkcí a vlastností aplikace při provádění změn, například rozšiřování jiných oblastí či opravě chyb \cite{test-regresni}. Jak uvádí \cite{test-bdo}, je třeba nalézt pomyslný \enquote{rovnovážný bod}, který reprezentuje optimální poměr manuálních a automatizovaných testů vzhledem k ceně jejich vytvoření a počtu běhů testu -- cena vytvoření automatizovaného testu je vysoká, pokud poběží jednou, ale rychle se snižuje s počtem opakování, naproti tomu manuální testování je levné, ale s každým během se cena zvyšuje kvůli délce běhu testu.

Díky vytvoření automatizovaných testů lze aplikaci průběžně testovat díky průběžnému testování na integračním serveru při každé revizi \cite{test-kitner}, to umožní ještě rychlejší zpětnou vazbu, rychlejší vydávání verzí a spokojenost zákazníka \cite{test-perfecto}. Aplikace je totiž prakticky připravená na nasazení v každé revizi \cite{test-atlassian}.

\section{Struktura automatizovaných testů}

Automatizované testování prostupuje více úrovněmi softwarového projektu \cite{test-kitner2} a je důležité na počátku tvorby testů stanovit, kde se bude automaticky testovat \cite{test-kitner}. Jednou z častých strategií zejména agilních týmů je strategie \enquote{testovací pyramidy} \cite{test-smartbear}.

\subsection{Testovací pyramida}

Testovací pyramida rozděluje automatizované testy do několika oblastí \cite{test-smartbear}. Jak uvádí \cite{test-kitner2, test-smartbear}, obvykle by měly být jádrem automatizovaných testů unit testy, které ověřují, zda jednotlivé dílčí části kódu odpovídají požadavkům. Těchto testů by měla být většina. Následují testy komponent (např. přihlášení uživatele, tvorba účtu, objednávky) a integrační testy (ověří, že jednotlivé komponenty spolu interagují tak, jak bylo zamýšleno, např. data zákazníka jsou napříč celým procesem objednávky korektně přenášena). Pod vrcholem pyramidy se nalézají API testy a na vrcholu jsou UI testy (někdy nazývané end-to-end, funkční či e2e), které by měly typicky mít nejmenší podíl ze všech automatizovaných testů vzhledem k jejich náročnosti na tvorbu a křehkosti při změnách UI.

% todo obrazek z https://watirmelon.blog/testing-pyramids/ nebo https://smartbear.com/learn/automated-testing/what-is-automated-testing/ nebo https://martinfowler.com/bliki/TestPyramid.html nebo https://medium.com/@fistsOfReason/testing-is-good-pyramids-are-bad-ice-cream-cones-are-the-worst-ad94b9b2f05f nebo https://medium.com/@timothy.cochran/test-pyramid-the-key-to-good-automated-test-strategy-9f3d7e3c02d5

Konkrétní aplikace této strategie závisí na vlastnostech projektu, týmu a požadavcích. Přesto se pokusím uvést obecná doporučení různých autorů. Čím níže z pohledu test pyramidy se podaří vývojářům dostat, tím lépe pro budoucí práci -- lépe začít např. s API, než s UI -- případně lze pracovat ve více lidech na více úrovních paralelně \cite{test-kitner}. Pokud ale aplikace už běží a postrádá jakékoliv testy, je ideální začít s tvorbou testů na vrcholu pyramidy pro kritické klíčové části businessu \cite{test-atlassian}. Pro zbytek je vhodné využít nižší úrovně testů \cite{test-atlassian2}.

Mnoho organizací má i přes testovací pyramidu většinu automatizovaných testů postavených nad UI vrstvou \cite{test-devqa}, výhodou je dohled nad výsledným produktem, který je pak používán, nevýhodou křehkost a nesnadné dohledávání původu chyb zachycených při testech UI, v případě použití AJAX (Asynchronous JavaScript and XML) je také mnohem složitější vytvořit deterministické UI testy \cite{test-timothy}. Testovací pyramida je pak obrácená a obvykle se tomuto přístupu vzhledem k tvaru obrácené pyramidy říká \enquote{zmrzlinový kornout} \cite{test-mf1}. 

Při volbě správné části k testování je třeba řídit se tím, která část je pro business zákazníka důležitá \cite{test-kitner}, než pouze například co nejvyšším pokrytím kódu testy \cite{test-devqa}. V případě, že se UI testy použijí pro klíčové funkce aplikace, můžeme zajistit, že i přes případné drobnější chyby funguje ta nejdůležitější část aplikace korektně, a to včetně všech vrstev jako např. napojení na API, databázi ad., to umožní častější dodávání nových verzí zákazníkovi s větší jistotou fungování, to vše je ale třeba dělat s vědomím jednotlivých úrovní testovací pyramidy a důsledků použití UI testů \cite{test-novanet}. Je tedy vhodné ve vyšší vrstvě testovat dva typy průchodů -- nejhorší a nejideálnější -- a pro zbytek hraničních případů využít vrstvy nižší \cite{test-dzone}.

Strategie testovací pyramidy je často špatně interpretována a může například vyústit v napsání naprosto všech možných unit testů a poté přesunutí do vyšší vrstvy atd., což pro některé týmy může být vhodné, pro jiné ale zbytečně komplexní a drahé \cite{test-cucumber2}. Principem pyramidy je naznačit, že testů na vyšší vrstvě má být méně než na nižší \cite{test-cucumber2}. Ačkoliv se princip testovací pyramidy stává standardem pro agilní týmy, je třeba rozumět principům, se kterými tento přístup přichází a na základě zavést způsoby testování pro konkrétní projekt \cite{test-cucumber2}. Na závěr je třeba říci, že veškeré zmíněné termíny, názvy a definice nejsou rigorózní a často panují různé názory např. na rozdělení testovací pyramidy, definici UI testování (někdy synonymum end-to-end testování, někdo nikoliv z toho důvodu, že UI lze testovat i pomocí unit testů) a mnoho dalšího, při tvorbě testů v rámci softwarového produktu je tedy vždy třeba jasně vymezit a definovat rozsah testů, sjednotit terminologii a na všem se domluvit \cite{test-fowler}.

\subsection{Alternativní přístupy}

Alternativním přístupem pro testování klientské části je \enquote{testovací trofej} \cite{test-trophy}, která se skládá ze 4 částí, od té nejnižší: statické testy (založené na statickém typování a linterech, viz TODO), unit testy (cílené na kritické části aplikace), integrační testy (ověřující, že vše spolu korektně spolupracuje) a end-to-end funkční testy (simulace chování uživatele při důležitých průchodech aplikací prostřednictvím automatizovaného klikání -- tedy UI testy z testovací pyramidy). Jak ale opět uvádí \cite{test-roth}, tento přístup opět může pro některé týmy a aplikace být vhodný, pro některé nikoliv a bez dostatečné znalosti konkrétního kódu aplikace a problematiky může následování nejen tohoto vzoru vyústit ve slepou cestu.

\section{Metodiky a nástroje pro automatizované testování}
Svět testování je ovlivňován několika metodikami vývoje softwaru, především TDD (Test-Driven development) a BDD (Behaviour-Driven development) \cite{test-swtestinghelp1}. S těmito metodikami souvisí také různé nástroje pro automatizované testování podporující příslušný způsob vývoje, o nich se též zmíním.

\subsection{TDD}
Základním principem TDD je napsat nejprve testovací případy (přímo v programovacím jazyce) a až poté implementovat kód, který zařídí splnění těchto případů \cite{test-swtestinghelp2}. Výsledkem je vyšší kvalita a flexibilita kódu (tím pádem lze pak jednoduše provádět např. refaktoring) a také vysoké pokrytí kódu \cite{test-swtestinghelp2}. Nevýhodou TDD je, že změny ve fungování aplikace mohou mít velký dopad na testovací případy, testovacím případům také rozumí pouze lidé se znalostí programovacích jazyků \cite{test-swtestinghelp2}. Vzhledem k zaměření testů na konkrétní implementaci a nikoliv chování je ale snadnější najít v případě TDD testů konkrétní chybu  \cite{test-swtestinghelp2}. TDD testy se zaměřují zejména na nejnižší vrstvu v testovací pyramidě, tedy unit testy \cite{test-swtestinghelp1}, principy TDD se ale mohou aplikovat napříč všemi vrstvami \cite{test-dzone}.

Mezi nástroje pro podporu TDD patří xUnit frameworky, např. JUnit, TestNG, NUnit \cite{test-swtestinghelp2}. Pro JS se dle \cite{test-chart-js2} a \cite{test-chart-js1} nejvíce používá Jest a Mocha, pro Python se používá PyTest nebo unittest \cite{test-chart-python}.

\subsection{BDD}
BDD rozšiřuje TDD a místo psaní testovacích případů se zapisují požadované scénáře chování, později je opět implementován samotný kód aplikace zařizující její předem specifikované chování \cite{test-swtestinghelp2}. Díky zápisu scénářů v jazyce Gherkin, který vychází z přirozeného jazyka, je možná jednoduchá spolupráce mezi vývojáři, testery, analytiky a zákazníkem (i bez znalosti programovacích jazyků) \cite{test-swtestinghelp2, test-cucumber1}, vytvořené scénáře tvoří prakticky základ pro dokumentaci funkcí aplikace \cite{test-smartbear2}. Při vývoji aplikace totiž často dochází k nedorozuměním ohledně výkladu různých požadavků. Výhodou BDD je také zaměření na samotné chování aplikace, nikoliv na přemýšlení o implementaci v kódu, chování aplikace je hlavní prvek, na který se vývojáři a testeři zaměřují a umožňuje to tak držet se lépe požadavků zákazníka \cite{test-swtestinghelp2}. BDD testy se zaměřují zejména na prostřední vrstvy v testovací pyramidě, tedy API testy \cite{test-swtestinghelp1}.

Mezi nástroje pro podporu BDD patří frameworky jako SpecFlow, Cucumber, MSpec \cite{test-swtestinghelp2}. Ze scénářů lze vytvářet vždy aktuální dokumentaci a reporty díky nástrojům jako TestComplete \cite{test-smartbear2}. Pro JS se používá Cucumber.js, pro Python se používá Behave \cite{test-chart-bdd}.

BDD se často používá spolu s TDD, protože dokáže na vyšší úrovni prověřit korektní fungování aplikace a poskytnout tak vyšší důvěru ve výsledný produkt \cite{test-cucumber1} -- takové testy tedy prověří nějaké chování aplikace a doplněním o TDD testy na nižších úrovních se dotestují specifické části.

\subsection{Další nástroje}

Jak jsem již zmínil, BDD se zaměřuje na vyšší vrstvy testovací pyramidy, pro UI testování se tedy často používá s nástroji jako Selenium \cite{test-dzone}. Selenium je framework, který prostřednictvím tzv. WebDriveru, který je implementován jednotlivými tvůrci prohlížečů umožňuje interagovat s webovou stránkou v prohlížeči (a to nejen v běžném módu, ale také tzv. \enquote{headless} módu, kde prohlížeč běží bez GUI) \cite{test-hackernoon1}. Jedná se o mocný nástroj, nevýhodou je ale poměrně zdlouhavý zápis skriptů pro výběr elementů, práci s výjimkami a časováním, proto se často volí nádstavby zaobalující Selenium, které umožňují jednodušší zápis, např. Selenide \cite{test-hackernoon1}. 

Co se týče \enquote{headless} módu prohlížečů, do nedávné doby byl nejpopulárnější volbou PhantomJS, vzhledem k postupné implementaci této funkce do předních webových prohlížečů byl jeho vývoj zastaven \cite{test-fowler, test-phantomjs}.

Selenium se postupem času stalo průmyslovým standardem pro UI testování, a to zejména díky jednoduchosti použití, kompatibilitě (s mnoha prohlížeči, možnost použití s mnoha programovacími jazyky) a popularitě \cite{test-selenium1}. Jak ale uvádí \cite{test-cypress1}, v případě asynchronních aplikací je Selenium složitější používat vzhledem k principu jeho fungování, protože je třeba vždy explicitně čekat na objevení elementu na stránce, problémem ale je, že nevíme, jak dlouho čekat. I z toho důvodu zde autor nabízí alternativu v podobě novějšího Cypress, který nabízí mnohem lepší práci s asynchronními aplikacemi, problémem ale je možnost použití pouze s JS, malá komunita a popularita, méně návodů \cite{test-cypress1}. Další nevýhodou Cypress byla podpora výhradně prohlížeče Google Chrome, to se ale v únoru 2020 změnilo \cite{test-cypress2}, naopak výhodou uváděnou např. v \cite{test-cypress3} byla lepší dokumentace oproti Seleniu, dokumentace Selenia ale byla kompletně přepsána a na podzim 2019 nasazena \cite{test-selenium2}.

Alternativ je ale mnohem více, z mnohých používaných například Katalon Studio či Robot framework \cite{test-selenium3}.

\chapter{Konfigurace více prostředí}
tudle kapitolu možná nezařadím a rovnou popíšu způsob řešení v praktické části
obsah: Deployment environments a s tím související Release management

\begin{itemize}
\item přehled toho, jak se můžou řešit různý prostředí, např. staging, testing, produkce apod., asi se tomu rika deployment pipeline
\end{itemize}
\begin{itemize}
\item přehled toho jak s tím souvisí releasy, co kam může jít a k čemu je to dobrý, jak se to dá různě dělat a souvislosti (např. nemusí to dělat travis na základě releasu, ale třeba na základě podmínky proměnné prostředí nebo branch production apod.)f
\end{itemize}

viz:

\begin{itemize}
\item https://continuousdelivery.com/implementing/patterns/\#the-deployment-pipeline
\item https://en.wikipedia.org/wiki/Deployment\_environment
\item https://docs.gitlab.com/ee/ci/environments.html
\item https://cs.wikipedia.org/wiki/Release\_management
\item https://docs.microsoft.com/cs-cz/aspnet/web-forms/overview/deployment/configuring-server-environments-for-web-deployment/scenario-configuring-a-staging-environment-for-web-deployment
\item http://guides.beanstalkapp.com/deployments/best-practices.html
\item ...
\end{itemize}

\chapter{Nástroje pro usnadnění vývoje a údržby}
rešerše nástrojů pro:
\begin{itemize}
\item odchytavani chyb (sentry...)
\item logovani (logentries...)
\item staticka analyza a kontrola kodu, hledani chyb, zero-days zranitelnosti, hodnoceni kvality kodu (LGTM...)
\item code formattery (prettier, black...) + souvislost s flake8 apod.
\item analyza pruchodu aplikaci (google analytics)
\item pokryti kodu (codecov, coverage.py...)
\item pripadne nejake pokrocile veci z travisu co se pouzivaji
\item lintery (ESlint.. + styly airbnb, standardjs, info o spouste pluginech, standardech, vyvoji ve spolupraci s TS)
\item static type checking pro frontend i backend (python, js) - moznosti, ze python ma builtin, JS nema (uvest proc, viz ecma vyjadreni), jak se to v JS resi a srovnat: FlowJS, Typescript + jak jsou na tom ostatni jako coffeescript, dart, kotlin..; u TS zminit podporu babel
\item + hledani dead code - python vulture
\item nastroje pro vyvoj frontendu - bakalarka byla postavena na CRA (create-react-app), ktera byla ejectnuta (takze update reactu byl prakticky nemozny), z toho duvodu se nejprv preslo na nastroj nwb, ktery ale pozdeji nebyl udrzovany a doslo ke zkusebni migraci na neutrinojs (souvisi s mozillou) a na vlastni reseni pres webpack - tady bych teda prosel, co tyhle nastroje umoznuji, ze vlastne stavi nad webpackem, ze je to vlastne abstraktni vrstva nad nim, ktera vsechno zjednodusuje, ale nevyhoda je, kdyz ten nastroj neco neumoznuje (viz nwb), tak mas smulu.. takze pak nejdou lintery, nejde TS... zminit moznosti nastroju, jake nastroje jsou, vyhody a nevyhody; v ramci implementace pak zminim cely pribeh s nwb, neutrinojs a custom webpackem, na zaver uvedu proc jsem vse zmigroval na custom webpack ikdyz neutrinojs fungovalo taky
\end{itemize}


\chapter{Zvolené technologie}
Na základě rešerše zvolím technologie, které pak zakomponuji v implementaci do appky - oargumentuji volbu..

tzn:

\begin{itemize}
\item 1. cim testovat a jak
\item 2. jaka prostredi a jak delat releasy, jak to vse bude fungovat (a fakt funguje)
\item 3. zvolene nastroje pro usnadneni vyvoje a udrzby
\end{itemize}