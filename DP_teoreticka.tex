\chapter{Aktuální řešení}
% todo citace, odkazy, lepsi popisy technologii (napr webpack)
V této kapitole stručně popíši verzi aplikace, která byla vytvořena v rámci bakalářské práce \cite{bp} a kterou budu v praktické části rozšiřovat. Popíši implementované funkce, použité technologie a také konfiguraci prostředí a způsob nasazování.

\section{Implementované požadavky}
Nejprve shrnu implementované funkční, poté i nefunkční požadavky \cite{bp}:
\subsection{Implementované funkční požadavky}
%TODO odkaz na obrazek puvodniho datoveho modelu
\begin{itemize}
    \item \textbf{evidence klientů},
    \item \textbf{evidence lekcí klientů:} lekce patří k nějakým kurzům, jsou pro jednotlivce nebo pro skupiny,
    \item \textbf{evidence údajů o lekci:} základní informace o lekci včetně stavů účasti všech účastníků a stavu jejich platby za danou lekci,
    \item \textbf{evidence předplacených lekcí:} předplacená lekce je řešená jako lekce, která nemá datum a čas konání),
    \item \textbf{evidence kurzů a stavů účasti na lekci},
    \item \textbf{přehled lekcí pro aktuální den},
    \item \textbf{karta klienta a skupiny}: obsahuje všechny informace o klientovi/skupině včetně všech lekcí,
    \item \textbf{upozornění na platbu příště:} když už klient nemá žádné předplacené lekce,
    \item \textbf{pořadové číslo lekce:} o kolikátou navštívenou lekci v pořadí se jedná,
    \item \textbf{týdenní přehled:} zobrazení lekcí jako v diáři.
\end{itemize}

\subsection{Implementované nefunkční požadavky}
\begin{itemize}
    \item \textbf{kompatibilita s běžnými webovými prohlížeči} v posledních verzích,
    \item \textbf{podpora široké škály zařízeních:} aplikace ke responzivní a korektně se zobrazuje na všech zařízeních používaných lektorkou -- iPad s iOS~11.3 a 9,7palcovým displejem, Nokia~5 s Androidem~8.0 a 5,2palcovým~displejem, notebook s rozlišením 1920~×~1080 a 15,6palcovým~displejem,
    \item \textbf{připravenost na rozšíření a údržbu}: kód tvořen s důrazem na budoucí možná rozšíření, použití konstant, respektování principu DRY,
    \item \textbf{bezpečnost:} JWT autentizace a další konfigurace pro zabezpečení aplikace,
    \item \textbf{srozumitelné a jednoduché rozhraní aplikace:} díky iterativnímu přístupu k návrhu UI a neustálé spolupráci s lektorkou.
\end{itemize}

\section{Použité technologie}
Serverová část aplikace je napsána v Pythonu 3.6.5 s webovým frameworkem Django 2.0.5. Pro správu závislostí se používá holý soubor \verb|requirements.txt|. Aplikace vystavuje REST API postavené na frameworku Django REST framework 3.8.2. Na produkci se používá webový server \href{http://gunicorn.org/}{Gunicorn 19.8.1} spolu s knihovnou WhiteNoise 3.3.1 pro efektivní servírování zkomprimovaných souborů.

Klientská část aplikace je napsána v JavaScriptu. Jedná se o SPA aplikaci postavenou na knihovně React 16.3 spolu s UI frameworkem \href{https://getbootstrap.com}{Bootstrap 4.1} a související knihovnou Reactstrap 5.0. Pro asynchronní požadavky na REST API využívá knihovnu axios 0.18. Celá klientská část je postavená na nástroji Create React App 1, který umožňuje \cite{cra} vytvářet React aplikace bez počáteční konfigurace, vzhledem k použití Djanga bylo ale potřeba \cite{bp} pomocí příkazu \verb|eject| \enquote{vysunout} celou konfiguraci a pomocí knihoven \href{https://github.com/owais/webpack-bundle-tracker}{webpack-bundle-tracker} a  \href{https://github.com/owais/django-webpack-loader}{django-webpack-loader} propojit Django s nástrojem Webpack.

V případě vývoje na lokálním stroji se používá pro serverovou část vývojový Django server a pro klientskou část WebpackDevServer, oba nabízejí podporu pro hot reloading a v tomto ohledu bylo vše i díky zmíněným knihovnám zprovozněno.

\section{Prostředí, testování a nasazování}
Aplikace je verzována v privátním repozitáři na serveru GitHub. Při každém nahrání nové revize na server (\verb|push|) se na integračním serveru Travis CI spustí sestavení aplikace, vytvoří se testovací databáze a spustí se základní testy (viz níže). Výsledné pokrytí kódu se poté nahraje na platformu \href{https://codecov.io/}{codecov.io} pro pokročilé statistiky o testování. Pokud vše na Travisu proběhne v pořádku, začne nasazení na produkční server běžící na Heroku. Nasazení na Heroku probíhá tak, že se zde opět spouští celé sestavení aplikace znovu, zmigruje se databáze a aplikace se nasadí. I tento průběh lze sledovat přímo z Travis terminálu.

Spouštěné testy jsou základní a velmi jednoduché, otestují:
\begin{itemize}
    \item přidání klienta a uživatele do databáze přes Django modely,
    \item zda Django uživateli zobrazí správnou stránku při příchodu do aplikace (neřeší se, zda se pak vůbec JS vyrenderuje),
    \item funkčnost API požadavků -- proběhne autorizace (a tedy získání JWT tokenu) a pokus o vytvoření klienta přes API.
\end{itemize}

Jak je vidět, testy byly skutečně pouze velmi povrchní, dalo by se říci, že se jedná o smoke testy. Také je třeba zdůraznit, že provedení \verb|push| na repozitář mělo za následek okamžité nasazení na produkci, pokud sestavení a tyto základní testy prošly. To může být nedozírné následky. I proto se v dalších kapitolách této teoretické části budu zaobírat možnostmi zlepšení.

\section{Plán rozšíření z bakalářské práce}
Na závěr bakalářské práce bylo zmíněno několik možných rozšíření aplikace. Následuje jejich stručný přehled, v kapitole~\ref{pozadavkyanalyza} se mimo jiné na některá z nich také dostane.
Přehled možných rozšíření \cite{bp}:
\begin{itemize}
    \item \textbf{vylepšení předplacených kurzů:} pohodlnější způsob předplacených lekcí pro skupiny (v aplikaci to sice možné bylo, ale velmi krkolomně a nepohodlně),
    \item \textbf{kontrola časového konfliktu lekcí},
    \item \textbf{vyhledávání v aplikaci},
    \item \textbf{evidování zájemců o kurz:} pro plánování nových lekcí kurzů a skupin,
    \item \textbf{evidence pomůcek a učebnic},
    \item \textbf{testy:} doplnění dalších testů a vysoké pokrytí kódu,
    \item \textbf{migrace na nový React:} k vydání došlo ke konci bakalářské práce, např. došlo ke změnám v API (životní cyklus komponent),
    \item \textbf{React Context API:} analýza možností využití Context API v rámci nového Reactu, zejména by pravděpodobně pomohlo snížit např. počet přístupů do API z klientské části napříč aplikací,
    \item \textbf{offline přístup:} analýza možností řešení offline přístupu, např. automatické ukládání do Google kalendáře, progresivní webové aplikace apod. a s tím související další oblasti jako SSR.
\end{itemize}

\chapter{Možnosti automatizovaného testování}

\begin{itemize}
\item popsat ruzne možnosti testování - BDD, TDD...
\end{itemize}
\begin{itemize}
\item ruzny scope - API, UI, unit.. vyhody, nevyhody...
\end{itemize}
\begin{itemize}
\item popsat selenium a alternativy (capybara?...)
\end{itemize}


\chapter{Konfigurace více prostředí}
obsah: Deployment environments a s tím související Release management


\begin{itemize}
\item přehled toho, jak se můžou řešit různý prostředí, např. staging, testing, produkce apod.
\end{itemize}
\begin{itemize}
\item přehled toho jak s tím souvisí releasy, co kam může jít a k čemu je to dobrý, jak se to dá různě dělat a souvislosti (např. nemusí to dělat travis na základě releasu, ale třeba na základě podmínky proměnné prostředí nebo branch production apod.)
\end{itemize}

viz:

\begin{itemize}
\item https://en.wikipedia.org/wiki/Deployment\_environment
\item https://docs.gitlab.com/ee/ci/environments.html
\item https://cs.wikipedia.org/wiki/Release\_management
\item https://docs.microsoft.com/cs-cz/aspnet/web-forms/overview/deployment/configuring-server-environments-for-web-deployment/scenario-configuring-a-staging-environment-for-web-deployment
\item http://guides.beanstalkapp.com/deployments/best-practices.html
\item ...
\end{itemize}

\chapter{Nástroje pro usnadnění vývoje a údržby}
rešerše nástrojů pro:
\begin{itemize}
\item odchytavani chyb (sentry...)
\item logovani (logentries...)
\item staticka analyza a kontrola kodu, hledani chyb, zero-days zranitelnosti, hodnoceni kvality kodu (LGTM...)
\item code formattery (prettier, black...) + souvislost s flake8 apod.
\item analyza pruchodu aplikaci (google analytics)
\item pokryti kodu (codecov, coverage.py...)
\item pripadne nejake pokrocile veci z travisu co se pouzivaji
\item lintery (ESlint.. + styly airbnb, standardjs, info o spouste pluginech, standardech, vyvoji ve spolupraci s TS)
\item static type checking pro frontend i backend (python, js) - moznosti, ze python ma builtin, JS nema (uvest proc, viz ecma vyjadreni), jak se to v JS resi a srovnat: FlowJS, Typescript + jak jsou na tom ostatni jako coffeescript, dart, kotlin..; u TS zminit podporu babel
\item + hledani dead code - python vulture
\item nastroje pro vyvoj frontendu - bakalarka byla postavena na CRA (create-react-app), ktera byla ejectnuta (takze update reactu byl prakticky nemozny), z toho duvodu se nejprv preslo na nastroj nwb, ktery ale pozdeji nebyl udrzovany a doslo ke zkusebni migraci na neutrinojs (souvisi s mozillou) a na vlastni reseni pres webpack - tady bych teda prosel, co tyhle nastroje umoznuji, ze vlastne stavi nad webpackem, ze je to vlastne abstraktni vrstva nad nim, ktera vsechno zjednodusuje, ale nevyhoda je, kdyz ten nastroj neco neumoznuje (viz nwb), tak mas smulu.. takze pak nejdou lintery, nejde TS... zminit moznosti nastroju, jake nastroje jsou, vyhody a nevyhody; v ramci implementace pak zminim cely pribeh s nwb, neutrinojs a custom webpackem, na zaver uvedu proc jsem vse zmigroval na custom webpack ikdyz neutrinojs fungovalo taky
\end{itemize}


\chapter{Zvolené technologie}
Na základě rešerše zvolím technologie, které pak zakomponuji v implementaci do appky - oargumentuji volbu..

tzn:

\begin{itemize}
\item 1. cim testovat a jak
\item 2. jaka prostredi a jak delat releasy, jak to vse bude fungovat (a fakt funguje)
\item 3. zvolene nastroje pro usnadneni vyvoje a udrzby
\end{itemize}