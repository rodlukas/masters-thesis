Proces vývoje jakékoliv aplikace nikdy nekončí a jinak tomu není ani v případě současné webové aplikace pro projekt \enquote{Úspěšný prvňáček}\footnote{\url{https://uspesnyprvnacek.cz/}} (dále jen ÚP) vedený speciální pedagožkou PaedDr. Janou Rodovou. Uplynuly přesně dva roky od nasazení první produkční verze v rámci mé bakalářské práce. Tomu předcházela dlouhá cesta příprav, analýz, rešerší, návrhů a poté implementace. Cesta to byla trnitá, i díky volbě mě naprosto cizích nejmodernějších technologií pro serverovou i klientskou část, snaze o jejich propojení, někdy nepříliš přívětivé dokumentaci či odlišným přístupům frameworků. Vše se ale nakonec zdárně podařilo a po akceptačním testování a posledním vyladění lektorka mohla spokojeně začít aplikaci každodenně používat. Existence této aplikace znamenala obrovský skok kupředu vzhledem k tomu, že do té doby byl užíván pro část evidence naprosto nedostačující Microsoft Excel, pro část diář a pro další část různé papíry. To vše díky aplikaci skončilo.

Na závěr bakalářské práce jsem uváděl možná rozšíření v budoucnu, a to nezůstalo jen u slov. O měsíc později po odevzdání bakalářské práce naplno začala druhá etapa vývoje aplikace. A o této etapě budu psát v rámci celé této diplomové práce. Během tohoto necelé dva roky trvajícího postupného vývoje bylo třeba řešit mnoho \enquote{kostlivců}, mnoho chyb, o kterých jsem v době vytváření neměl ani tušení. Ale také mnoho nových funkcí, protože bylo potřeba jít kupředu vzhledem k rozvoji samotného ÚP. Jak vývoj postupoval, přicházely další a další nové výzvy a poznání. Objevilo se také několik slepých uliček. Ale nepředbíhejme.

U projektu ÚP jsem již od samého počátku, a proto je tato webová aplikace mou srdeční záležitostí. Mojí snahou bylo tedy vyhovět naprosto všem požadavkům lektorky tak, aby s každou novou verzí byla její práce hladší, jednodušší a efektivnější. Další motivací je možnost si v rámci tohoto vývoje vyzkoušet další řadu nových technologií, nástrojů a přístupů a zdokonalit se tak v oblasti tvorby moderních webových aplikací.

V teoretické části nejprve stručně popíši, co nabízí původní verze aplikace -- jaké má funkce, použité technologie, jak funguje nasazování a do jakých prostředí se nasazuje. Zmíním zde i stručně původní náměty na rozšíření aplikace do budoucna, o kterých bylo uvažováno na závěr bakalářské práce. Poté se zaměřím na možnosti automatizovaného testování webových aplikací, způsoby konfigurace více prostředí pro nasazování a možné nástroje pro usnadnění vývoje a údržby. Na závěr zvolím vhodné možnosti a nástroje pro tuto práci a volbu oargumentuji.

V praktické části nejprve provedu sběr a analýzu požadavků a navrhnu úpravy včetně aktualizovaného databázového modelu a upraveného schéma API (Application Programming Interface). Poté již popíši postupně všechny kroky v rámci implementace všech požadavků a zavedení nástrojů. Následovat bude podrobný popis vytváření testů, které budou tvořit důležitý podpůrný prvek celého vývoje a popis nového způsobu nasazování aplikace včetně konfigurace více prostředí. Na závěr uvedu možná rozšíření aplikace v budoucnu a také popíši kroky učiněné pro zveřejnění aplikace jako open-source.