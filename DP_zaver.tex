Úspěšně jsem rozšířil webovou aplikaci o všechny požadavky lektorky. Umožňuje tak nově např. evidovat zájemce o kurz či pracovat mnohem jednodušeji s předplacenými lekcemi. Aplikace je lépe použitelná, díky mnohým vylepšením se dá mnohem efektivněji používat, díky zavedeným kontrolám omezení dbá na korektní a konzistentní data (např. časové konflikty lekcí). Zavedl jsem také další zvolené nástroje pro usnadnění vývoje a údržby, dokumentaci v kódu, automatizované testy API a UI (e2e) a nakonfiguroval jsem nasazování do více prostředí. Díky tomu je možné spolehlivější a rychlejší dodávání nových verzí a na fungování aplikace se dá více spolehnout, mnoho chyb bylo odstraněno a mnohým je díky testům předcházeno. Aplikace je také díky pokročilé optimalizaci mnohem rychlejší a splňuje více bezpečnostních standardů.

Vývoj probíhal v rámci iterací po celé dva roky a tato práce je jakýmsi završením a shrnutím mnoha učiněných kroků (nikoliv samozřejmě všech). Aplikace je v projektu úspěšně každodenně používána od května~2018 a už od svého počátku lektorce výrazně usnadnila práci a ušetřila mnoho času. Díky všem zavedeným změnám aplikace pokrývá více oblastí projektu a lépe řeší mnoho již pokrytých oblastí a umožňuje tak práci zrychlit a usnadnit ještě více.

Taktéž se lze zpětně ohlédnout za samotnou podobou návrhu a implementace původní verze z bakalářské práce. Vzhledem k tomu, jak jednoduché bylo aplikaci rozšířit, lze říci, že použité technologie a návrh byl proveden dobře, původní implementace vyžadovala zásahy např. v podobě refaktoringu, ale z obecného pohledu vzhledem ke snadnému rozšiřování v rámci této práce zde taktéž nebyly větší problémy. Aplikace díky všem provedeným krokům v rámci této práce dospěla a je dobře připravena pro další údržbu a případné rozšiřování.

Aplikace je nyní nasazená na všechny prostředí včetně produkce a lektorka ji spokojeně každý den používá. 
